\documentclass{udpreport}
%\headertext{Metodos Numéricos}
\title{Metodos Numéricos : Tarea 1}
\author{Andrés Comte, Benjamín Morales, Dagoberto Navarrete.}
\usepackage{amssymb}
\usepackage{amsmath}
\usepackage{graphicx}
\usepackage{float}
\usepackage{array}
\graphicspath{ {Imagenes/} }
\usepackage{listings}
\usepackage{color}

\definecolor{dkgreen}{rgb}{0,0.6,0}
\definecolor{gray}{rgb}{0.5,0.5,0.5}
\definecolor{mauve}{rgb}{0.58,0,0.82}

\lstset{frame=tb,
  language=MATLAB,
  aboveskip=3mm,
  belowskip=3mm,
  showstringspaces=false,
  columns=flexible,
  basicstyle={\small\ttfamily},
  numbers=none,
  numberstyle=\tiny\color{gray},
  keywordstyle=\color{blue},
  commentstyle=\color{dkgreen},
  stringstyle=\color{mauve},
  breaklines=true,
  breakatwhitespace=true,
  tabsize=4
}

\begin{document}
\maketitle
\tableofcontents
\listoffigures

\chapter{Introducción}

.
\chapter{Resolución de ecuaciones no lineales:}
\begin{enumerate}
\item Usando los métodos de bisección, falsa posición, y secante, encuentre la raíz aproximada 
de las siguientes ecuaciones no lineales en los intervalos indicados:

    \begin{enumerate}
        
    Para realizar estas ecuaciones se utilizaron los programas de :
    \begin{itemize}
        \item Biseccion.m
        \item Secante.m
        \item Falsa.m
    \end{itemize}
    y las comparaciones de error se hicieron en base a el resultado de la función fzero de matlab
    
    \item  \(x^3 - 4sen(x) +1 = 0\) , sobre [0,1] , [1,2].
        \begin{figure}[H]
            \centering
            \includegraphics[width=9cm]{ec1.png}
             \caption{Gráfico Ej 1A}
        \end{figure}
        
            \begin{table}[H]Intervalo [0,1]:
            \centering
                \begin{tabular} { |c|c|c|c|}
                \hline
                Métodos       & Bisección & Falsa Posición & Secante  \\
                \hline
                Cero Obtenido &  0,2571       &    0,2571      &      0,2571    \\
                \hline
                Iteraciones   &    17        &     4     &       7        \\
                \hline
                Error Obtenido(\%) &       0      &       0      &     0         \\
                \hline
                Tiempo CPU &       0,009439     &      0,007311     &     0,002157         \\
                 \hline
                \end{tabular}
            \end{table}
            
        \begin{table}[H]Intervalo [1,2]:
        \centering
           \begin{tabular} { |c|c|c|c|}
                \hline
                Métodos       & Bisección & Falsa Posición & Secante  \\
                \hline
                Cero Obtenido &  1,4365       &    1,4364      &      1,4365   \\
                \hline
                Iteraciones   &    17        &    14     &       7        \\
                \hline
                Error Obtenido(\%) &       0      &       0,00696      &     0         \\
                \hline
                Tiempo CPU &       0,009656     &      0,016843    &     0,002164         \\
                 \hline
                \end{tabular}
            \end{table}
    
        
    \item \( e^{-t/2} +cos(2t) = 0 \), sobre [0,1].
    
        \begin{figure}[H]
            \centering
            \includegraphics[width=8cm]{ec2.png}
            \caption{Gráfico Ej 1B}
        \end{figure}
        
        \begin{table}[H]
        \centering
           \begin{tabular} { |c|c|c|c|}
                \hline
                Métodos       & Bisección & Falsa Posición & Secante  \\
                \hline
                Cero Obtenido &  1.0940      &    1.0940     &      2,1852   \\
                \hline
                Iteraciones   &    14        &    4     &      7        \\
                \hline
                Error Obtenido(\%) &       0      &       0      &     99,74         \\
                \hline
                Tiempo CPU &       0,008504     &      0,006189    &     0,002032         \\
                 \hline
                \end{tabular}
            \end{table}
      
        
    
    \item \(x + 50 -x\cosh(\frac{60}{x}) = 0 \), sobre [40,55].\\
        \begin{figure}[H]
            \centering
            \includegraphics[width=10cm]{ec3.png}
            \caption{Gráfico Ej 1C}
        \end{figure}
            \begin{table}[H]
            \centering
           \begin{tabular} { |c|c|c|c|}
                \hline
                Métodos       & Bisección & Falsa Posición & Secante  \\
                \hline
                Cero Obtenido &  42,4175      &    42,4175     &      42,4175   \\
                \hline
                Iteraciones   &    19        &    5     &      6        \\
                \hline
                Error Obtenido(\%) &       0      &       0      &     0         \\
                \hline
                Tiempo CPU &       0,009362     &      0,007245    &     0,002157         \\
                 \hline
                \end{tabular}
            \end{table}
      

    
    \item \(x^3-e^(-x)+xsen(3x) = 0 \), sobre [0,2].\\
        
        \begin{figure}[H]
        \centering
        \includegraphics[width=10cm]{ec4.png}
        \caption{Gráfico Ej 1D}
        \end{figure}
        \begin{table}[H]
        \centering
           \begin{tabular} { |c|c|c|c|}
                \hline
                Métodos       & Bisección & Falsa Posición & Secante  \\
                \hline
                Cero Obtenido &  0,493      &    0,493     &      0,493   \\
                \hline
                Iteraciones   &   17        &    16     &      6     \\
                \hline
                Error Obtenido(\%) &       0      &       0      &     0         \\
                \hline
                Tiempo CPU &      0,008646     &      0,017988    &     0,001805         \\
                 \hline
                \end{tabular}
            \end{table}
      
    \newpage
    
    
    \end{enumerate}
\newpage
\item Considere la ecuación no lineal $f(x)= \frac{1}{2}+\frac{1}{4}x^2-xsen(x)-\frac{1}{2}cos(2x)=0$
    	\begin{enumerate}
    	\item  Usando el método de Newton con punto inicial $x_{0}=\frac{\pi}{2}$ encuentre la solución aproximada de $f$. Para ello realize las iteraciones necesarias hasta que se cumpla el criterio de parada $ |x_{n+1} - x_{n} < 10^{-6} $
    	\begin{figure}[H]
        \centering
        \includegraphics[width=10cm]{ec5.png}
        \caption{Gráfico Ej 1D}
        \end{figure}
        	\begin{table} [H]
        			\centering
        			\begin{tabular}{|c|c|c|}
        				\hline
        				$x_{0}$ & Cero Obtenido & Iteraciones\\
        				\hline
        				$\frac{\pi}{2} $ &  2.7904*10^(-6) & 56\\
        				\hline 
        			\end{tabular}
        		\end{table}
        	
        	\item Repetir el proceso con el método de Newton modificado, esto es, con la iteración $$x_{n+1} = x_{n} - \frac {f(x_{n})} {f'(x_{0})} $$
        	¿Que método converge mas rápido ?
        \begin{table} [H]
        			\centering
        			\begin{tabular}{|c|c|c|}
        				\hline
        				$x_{0}$ & Cero Obtenido & Iteraciones\\
        				\hline
        				$\frac{\pi}{2} $ &  0.0025 & 2512\\
        				\hline 
        			\end{tabular}
        		\end{table}
        \item Repita el proceso tomando como valores iniciales $x_{0}=5\pi$ y $x_{0}=10\pi$ ¿La sucesión construida con estos puntos iniciales converge?
        	 	\begin{table} [H]Newton 
        			\centering
        			\begin{tabular}{|c|c|c|}
        				\hline
        				$x_{0}$ & Cero Obtenido & Iteraciones\\
        				\hline
        				$5\pi$ & 1.8972 & 1665 \\
        				\hline 
        				$10\pi$ & 1.8972 & 1669\\
        				\hline
        			\end{tabular}
        		\end{table}
        		
        		\begin{table} [H]Newton Modificado
        			\centering
        			\begin{tabular}{|c|c|c|}
        				\hline
        				$x_{0}$ & Cero Obtenido & Iteraciones\\
        				\hline
        				$5\pi$ & 1.9003 & 4713 \\
        				\hline 
        				$10\pi$ & 1.9023 & 6620\\
        				\hline
        			\end{tabular}
        		\end{table}
        		
        \item Use el método de la secante para encontrar la solución aproximada tomando como puntos iniciales $x_{0}=5\pi$ y $x_{0}=10\pi$, como criterio de parada el mismo descrito en (a).
        	\begin{table} [H] 
        			\centering
        			\begin{tabular}{|c|c|c|c|}
        				\hline
        				$x_{0}$& $x_{1}$ & Cero Obtenido & Iteraciones\\
        				\hline
        				$\frac{\pi}{2}$ & 1.5415 & 1.8947 & 11 \\
        				\hline 
        				$5\pi$& 11.7810 & 1.8961     &16\\
        				\hline
        			\end{tabular}
        		\end{table}
        		
        		
        
       
        
    \end{enumerate}
		

\item El dinero necesario para pagar la cuota correspondiente a un crédito hipotecario a interés fijo se suele
estimar mediante la denominada “ecuación de la anualidad ordinaria”:
\begin{center}
    $ Q = \frac{A}{i}(1-(1+i)^{-n}) $
\end{center}
donde Q es la cantidad pedida en préstamo, A es la cuota que debe pagar el beneficiario por el
préstamo, i es la tasa de interés fijado por la entidad bancaria que concede el préstamo y n es el
número de periodos durante los cuales se realizan pagos de la cuota.
Una pareja que desea comenzar una vida en común se plantea adquirir una vivienda y para ello saben
que necesitan pedir un préstamo de 16000 dólares a pagar semestralmente durante un plazo de 10 años.
Sabiendo que para atender este pago pueden destinar una cantidad máxima de 400 dólares mensuales,
calcule cual es el tipo máximo de interés al que pueden negociar su préstamo con las entidades bancarias.¿Con dicho interés alguna de las entidades bancarias le concederá el préstamo solicitado?.
Hint.- Usar método de Newton, tomando como punto inicial i0 = 0.03.
Suponga ahora que desean endeudarse en 16 años en lugar de 10. Cual sería el interés en esta situación?


Resp: Al realizar el método de Newton aplicándolo a optimización, se obtiene un interes de 10.24\%.Si se Consideraba un error de 5\% el porcentaje de interes era excesivamente alto (orden del 40\%), es por ello que se realizarón iteraciones hasta observar que el valor no cambiaba, así se encontro el valor en el que converge a los 10 años del préstamo.

\begin{figure}[H]
    \centering
    \includegraphics[width=12cm]{1}
    \caption{Interés a los 10 años del préstamo}
\end{figure}
Suponga ahora que desean endeudarse en 15 años en lugar de 10 ¿Cual sera el interes en esta situacion?
\\
Resp: 5.1\% . El proceso de busqueda fue similar al anterior.
\\
\begin{figure}[H]
    \centering
    \includegraphics[width=12cm]{3}
    \caption{Interés a los 15 años del préstamo}
\end{figure}
\newpage


\item Considere la función \(f(x) = x*cos(x)-e^x+ 1\). % pregunta 5 
\begin{enumerate}
    
\vspace{0.9cm}
\item Considere las siguientes funciones. Realice unas 12 iteraciones de punto fijo, usando como puntos iniciales x0 = -0.5 y x0 = 0.5.\\ 


\begin{equation}
 g1(x) = \frac{e^x+x-1}
{1 + cos(x)}
\end{equation}
\\
\begin{itemize}
\item x0=0.5
\end{itemize}


\begin{table}[H]
    \centering
        \begin{tabular} { |c|c|}
        
        \hline
        iteración  &  Punto\\
        \hline
        1 &  0.6118        \\
         \hline
        2 &   0.8004       \\
         \hline
        3 &   1.1947       \\
         \hline
        4 &   2.5579      \\
         \hline
        5 &  87.3616      \\
         \hline
        6 & 4.7832e+37     \\
         \hline
        7 &  Inf         \\
         \hline
        8 &  NaN       \\
         \hline
        9 & NaN      \\
         \hline
        10 &  0 NaN       \\
         \hline
        11 &  NaN        \\
         \hline
        12 &   NaN        \\
        \hline
        
        \end{tabular}
    \end{table}
 
 \begin{itemize}
 \newpage
\item x0=-0.5
\end{itemize}
\begin{table}[H]
    \centering
        \begin{tabular} { |c|c|}
        
        \hline
        iteración  &  Punto\\
        \hline
        1 & -0.4759        \\
         \hline
        2 &  -0.4524        \\
         \hline
        3 &  -0.4298       \\
         \hline
        4 &   -0.4081      \\
         \hline
        5 &  -0.3875       \\
         \hline
        6 &   -0.3680     \\
         \hline
        7 &  -0.3497      \\
         \hline
        8 &  -0.3324       \\
         \hline
        9 &  -0.3163       \\
         \hline
        10 &  -0.3012       \\
         \hline
        11 &  -0.2871     \\
         \hline
        12 &  -0.2871       \\
        \hline
        \end{tabular}
\end{table}

\begin{equation}
 g2(x) = \frac{\sqrt{x(e^x-1)}}
{cos(x)}
\end{equation}
\\
\begin{itemize}
\item x0=0.5
\end{itemize}

\begin{table}[H]
    \centering
        \begin{tabular} { |c|c|}
        
        \hline
        iteración  &  Punto\\
        \hline
        1 &  0.6080       \\
         \hline
        2 &   0.7872     \\
         \hline
        3 &  1.1555       \\
         \hline
        4 &   2.4964     \\
         \hline
        5 & 0.0000 + 5.8994i        \\
         \hline
        6 &  0.1106 - 0.0106i       \\
         \hline
        7 & 0.1140 - 0.0113i         \\
         \hline
        8 &    0.1177 - 0.0121i     \\
         \hline
        9 &     0.1216 - 0.0130i     \\
         \hline
        10 &   0.1258 - 0.0140i       \\
         \hline
        11 &    0.1303 - 0.0151i   \\
         \hline
        12 &  0.1351 - 0.0163i        \\
        \hline
        
        \end{tabular}
        
    \end{table}
    \begin{itemize}
\item x0=-0.5
\end{itemize}

\begin{table}[H]
    \centering
        \begin{tabular} { |c|c|}
        
        \hline
        iteración  &  Punto\\
        \hline
        1 &  0.4735       \\
         \hline
        2 &   0.5676     \\
         \hline
        3 &  0.7171
       \\
         \hline
        4 &   0.9989     \\
         \hline
        5 & 1.7791        \\
         \hline
        6 &  0.0000 + 6.5089i      \\
         \hline
        7 &   0.0037 - 0.0660i       \\
         \hline
        8 &     0.0026 - 0.0660i   \\
         \hline
        9 &   0.0015 - 0.0660i       \\
         \hline
        10 &    0.0004 - 0.0660i      \\
         \hline
        11 &    0.0007 + 0.0659i \\
         \hline
        12 &   0.0004 - 0.0659i       \\
        \hline
        
        \end{tabular}
        
    \end{table}
 \vspace{3cm}   
 \item Teniendo en cuenta las siguientes funciones de iteración de punto fijo.  Realice unas 12 iteraciones de punto fijo, usando como puntos iniciales x0 = 0.5 y x0 = 0.5.\\   
 
 \begin{equation}
 g3(x) = x-\frac{f(x)}
{f'(x)}
\end{equation}
\\
 \begin{itemize}
\item x0=0.5
\end{itemize}

\begin{table}[H]
    \centering
        \begin{tabular} { |c|c|}
        
        \hline
        iteración  &  Punto\\
        \hline
        1 &   0.2923       \\
         \hline
        2 &     0.1645   \\
         \hline
        3 &  0.0891 \\
         \hline
        4 &  0.0468     \\
         \hline
        5 &    0.0241    \\
         \hline
        6 & 0.0122       \\
         \hline
        7 &    0.0062 \\
         \hline
        8 &  0.0031     \\
         \hline
        9 &      0.0015    \\
         \hline
        10 &    7.7566e-04     \\
         \hline
        11 &   3.8803e-04  \\
         \hline
        12 &   1.9407e-04      \\
        \hline
        
        \end{tabular}
        
    \end{table}
 \begin{itemize}
\item x0=-0.5
\end{itemize}

\begin{table}[H]
    \centering
        \begin{tabular} { |c|c|}
        
        \hline
        iteración  &  Punto\\
        \hline
        1 &  0.9462        \\
         \hline
        2 &     0.5755  \\
         \hline
        3 &    0.3398        \\
 
         \hline
        4 &   0.1932  \\
         \hline
        5 &  0.1057       \\
         \hline
        6 &      0.0560  \\
         \hline
        7 & 0.0289    \\
         \hline
        8 & 0.0147      \\
         \hline
        9 &   0.0074       \\
         \hline
        10 &    0.0037     \\
         \hline
        11 &   0.0019  \\
         \hline
        12 &   9.3724e-04     \\
        \hline
        
        \end{tabular}
        
    \end{table}
 \newpage
     \begin{equation}
 g4(x) = x-2\frac{f(x)}
{f'(x)}
\end{equation}
 \begin{itemize}
\item x0=0.5
\end{itemize}

\begin{table}[H]
    \centering
        \begin{tabular} { |c|c|}
        
        \hline
        iteración  &  Punto\\
        \hline
        1 &      0.0846   \\
         \hline
        2 &    0.0041    \\
         \hline
        3 &   1.1230e-05 \\
         \hline
        4 &   8.4360e-11    \\
         \hline
        5 &    8.4360e-11  \\
         \hline
        6 &   8.4360e-11     \\
         \hline
        7 & 8.4360e-11    \\
         \hline
        8 &  8.4360e-11   \\
         \hline
        9 &   8.4360e-11      \\
         \hline
        10 &   8.4360e-11      \\
         \hline
        11 & 8.8.4360e-11  \\
         \hline
        12 &  8.4360e-11      \\
        \hline
        
        \end{tabular}
        
    \end{table}
     \begin{itemize}
\item x0=-0.5
\end{itemize}

\begin{table}[H]
    \centering
        \begin{tabular} { |c|c|}
        
        \hline
        iteración  &  Punto\\
        \hline
        1 &     2.3924  \\
         \hline
        2 &    0.6344   \\
         \hline
        3 &   0.1196 \\
         \hline
        4 &    0.0078   \\
         \hline
        5 &   3.9737e-05  \\
         \hline
        6 &   1.0509e-09    \\
         \hline
        7 & 1.0509e-09   \\
         \hline
        8 &  1.0509e-09  \\
         \hline
        9 &  1.0509e-09      \\
         \hline
        10 &  1.0509e-09      \\
         \hline
        11 & 1.0509e-09   \\
         \hline
        12 &  1.0509e-09     \\
        \hline
        
        \end{tabular}
        
    \end{table}
    \newpage
    \begin{equation}
 g5(x) = x-\frac{f(x)f'(x)}
{(f'(x))^2-f(x)f''(x)}
\end{equation}
\\
 \begin{itemize}

\item x0=0.5
\end{itemize}
\begin{table}[H]
    \centering
        \begin{tabular} { |c|c|}
        
        \hline
        iteración  &  Punto\\
        \hline
        1 &  -0.0551      \\
         \hline
        2 &    -0.0023   \\
         \hline
        3 &  -3.5887e-06 \\
         \hline
        4 &  1.2915e-10    \\
         \hline
        5 &    2.5829e-10  \\
         \hline
        6 &  2.5829e-10     \\
         \hline
        7 &     2.5829e-10 \\
         \hline
        8 &  2.5829e-10    \\
         \hline
        9 &       2.5829e-10   \\
         \hline
        10 &    2.5829e-10     \\
         \hline
        11 &    2.5829e-10 \\
         \hline
        12 &   2.5829e-10  \\
        \hline
        
        \end{tabular}
        
    \end{table}
     \begin{itemize}
\item x0=-0.5
\end{itemize}
\begin{table}[H]
    \centering
        \begin{tabular} { |c|c|}
        \hline
        iteración  &  Punto\\
        \hline
        1 &  -0.4614      \\
         \hline
        2 &   -0.3825   \\
         \hline
        3 &  -0.2366 \\
         \hline
        4 &  -0.0667    \\
         \hline
        5 &   -0.0035 \\
         \hline
        6 & -8.1704e-06     \\
         \hline
        7 &     -2.6233e-11 \\
         \hline
        8 &   -2.6233e-11   \\
         \hline
        9 &        -2.6233e-11 \\
         \hline
        10 &     -2.6233e-11    \\
         \hline
        11 &    -2.6233e-11 \\
         \hline
        12 &    -2.6233e-11  \\
        \hline
        \end{tabular}
    \end{table}
\newpage
\item Que puede decir sobre el comportamiento de las iteraciones de punto fijo calculadas anteriormente.
\begin{itemize}

\item De la función g2(x), tanto para el punto inicial x0=0.5 como para x0=-0.5, desde determinada iteración la solución se volvió imaginaria, por lo tanto g2(x), al no ser una función continua, no tiene puntos fijos.
\item En cuanto a la función g3(x),para el punto inicial x0=0.5, la función cada vez va convergiendo a la solución, por lo que 0.5 es punto fijo de g3(x).\\ Ahora para x0=-0.5 al igual que en el anterior punto fijo, este igual converge a la solución, declarandose así que -0.5 tambien es un punto fijo de g3(x).
\item En cuanto a la función g4(x), para el punto x0=0.5 en las primeras iteraciones va convergiendo a la solución, hasta que se mantiene fijo en un valor. De esto se deduce que se encuentra la solución tomando como punto fijo 0.5.\\
Lo mismo sucedió con x0=-0.5 desde determinada iteración fue convergiendo hasta llegar a un valor fijo el cual no siguió variando, se deduce lo mismo que en el caso anterior, en el que a partir del punto fijo -0.5, se llega a la solución.
\item Acerca de la función g5(x) pasa algo similar a lo que sucedio con g4(x), en las cuales se encuentra la solución a partir de los puntos fijos tomados.
\end{itemize}

\end{enumerate}

\item {\bf Métodos iterativos:} A continuación se describen algunos métodos de iteración de punto fijo que encuentran la raíz aproximada de f(x) = 0
\begin{itemize}
        \item Método de Newton     
            \begin{equation*}
                 x_{n+1}=N_f(x_n)=x_n -  \frac{f(x_n)}{f^{'}(x_n)}
            \end{equation*}
        
        \item Método de Schröder
            \begin{equation*}
                x_{n+1}=S_f(x_n)=x_n-\frac{f(x_n)f^{'}(x_n)}{f^{'}(x_n)^{2}-f(x_n)f^{''}(x_n)}
            \end{equation*}
        
        \item Método de aceleración convexa de Whittaker
            \begin{equation*}
                x_{n+1}=W_f(x_n)=x_n - \frac{f(x_n)}{2f^{'}(x_n)}(2 - L_f(x_n))
            \end{equation*}
        
\end{itemize}
	\begin{enumerate}
	
		\item Utilice los métodos para encontrar la solución de la ecuación no lineal $f(x)=0$, donde
		
		(i) $f_{1}(x)=xe^{x^2}-sen(x)+4cos(x)+6, en [-2,0].$
		
		\begin{table}[H]
			\centering
			\begin{tabular}{|c|c|c|c|}
				\hline
				Métodos & $x_{0}$ & Iteraciones & Cero Obtenido \\
				\hline
				Newton & -2 &  7 & -1.3341\\
				\hline
				Schroder & -2 & 2 & NaN \\
				\hline
				Whittaker & -2 & 10 & -1.3341 \\
				\hline				
			\end{tabular}
			\end{table}	
		
		
		
		(ii) $f_{2}(x)=x^3-3x^2(2^{-x})+3x(4^{-x})-8^{-x}, en [0,1].$
		
			\begin{table}[H]
			\centering
			\begin{tabular}{|c|c|c|c|}
				\hline
				Métodos & $x_{0}$ & Iteraciones & Cero Obtenido \\
				\hline
				Newton & 0 & 44 &  0.6412 \\
				\hline
				Schroder & 0 & 4 & 0.6412\\
				\hline
				Whittaker & 0 & 68 & 0.6412\\
				\hline				
			\end{tabular}
			\end{table}	
			
		(iii) $f_{3}(x)=cos(\frac{x}{2}+\sqrt{2})+x(2x+5), en [-2,\frac{1}{2}].$	
		
		\begin{table}[H]
			\centering
			\begin{tabular}{|c|c|c|c|}
				\hline
				Métodos & $x_{0}$ & Iteraciones & Cero Obtenido \\
				\hline
				Newton & -2 & 4 &  -2.2895 \\
				\hline
				Schroder & -2 &  4 & -2.2895\\
				\hline
				Whittaker & -2 & 4 & -2.2895 \\
				\hline				
			\end{tabular}
		\end{table}		
\newpage			
		\item Compare los errores relativos vs. las iteraciones (mediante un gráfico) para los métodos (descritos
			anteriormente) que determinan la raíz de $f_{i}(x)=0$, para $i$ = 1,2,3.	
			\begin{itemize}
				\item Método de Newton:
				\begin{figure}[H]
					\centering	
						\includegraphics[width=10cm]{NewtonEj6} 
						\caption{Método de Newton}
				\end{figure}
								
				\item Método de Schröder: 
				 \begin{figure}[H]
					\centering
					\includegraphics[width=10cm]{SchroderEj6}
					\caption{Método de Schröder}
				\end{figure}	
				 
				 
				 \item Método de Whittaker;
				 \begin{figure}[H]
					\centering
						\includegraphics[width=10cm]{WhittakerEj6}
						\caption{Método de Whittaker}
				\end{figure}	
			\end{itemize}
			
			Al analizar la cantidad de iteraciones utilizadas por cada método, se puede concluir que el más eficiente es el método de Schroder.
			
			Para resolver este problema, se ocuparon los archivos newtonEj6.m, schroder.m y whittaker.m.
        \end{enumerate}
\end{enumerate}
\newpage
\chapter{Resolución de sistemas de ecuaciones no lineales:}
    
        
        En los siguientes ejercicios, use como criterio de parada lo siguiente:
        \begin{center}
            $ \frac{|| x^{n+1} - x^{n} ||}{|| x^{n+1} ||} \leq 10^{-6} $  
        \end{center}
        

        Donde $||z||=\sqrt{z_{1}^{2} + z_{2}^2+...+z_{n}^{2}} , z \in \mathbb{R}^{n}$
    \begin{enumerate}   
        
        \item Considere los siguientes sistemas de ecuaciones no lineales\\
        
        \begin{math}
            (a)\left\lbrace
          \begin{array}{ll}
                x^2 + xy^2-x-2&=0 \\\
                y^2 + xy^2-y+4&=0
            \end{array}
            \right.
            \hspace{3cm}
            (b) \left\lbrace
           \begin{array}{ll}
                 x^2+x-y^2&=1  \\
                 y-cos(x^2)&=0 
            \end{array}
            \right.
        \end{math}\\
        
        Usando el método de Newton, encuentre los puntos de intersección de las curvas descritas en los ítem (a) y (b). Para encontrar los puntos iniciales (x0,y0), apóyese en gráficos que le permitan encontrar una aproximación a la solución.
        
        
        Pregunta $ 1a) $\\
        
            
            $ Recta(1) : x^2+xy^2-x-2=0 $
            
            $ Recta(2) : y^2 +xy^2-y+2=0 $
            
            \begin{figure}[h]
                \centering
                \includegraphics[width=15cm]{GraficoEcEj1a}
                 \caption{Gráfico Ecuación Ej1A}
            \end{figure}
            
            Como se observa en el grafico las rectas (1) y (2) se intersectan 2 veces, mientras que luego tienden a alejarse una de la otra. \\Analizaremos las 2 soluciones que existen dentro de este intervalo de gráfica mostrado anteriormente. Luego, realizando un zoom en donde se encuentran las intersecciones, se obtiene lo siguiente.\\
            
            \begin{figure}[H]
               \centering   \includegraphics[width=13cm]{GraficoEcEj1azoom}
                \caption{Gráfico Ecuación Ej 1A con acercamiento}
            \end{figure}
            Ahora, para lograr obtener nuesta solución que se encuentra en el 2do cuadrante, ocuparemos un punto cercano a la interseccion. En este caso ocuparemos $ x0=[-2;1] $ y el resulado obtenido como primera solución de la ecuación es:
            $ x*=[-1.6088;1.1686 ]$ siendo el punto  $x = -1.6088$,$y = 1.1686$
            
            Como ya obtuvimos la primera solución, buscaremos la segunda, la cual se encuentra en el 3er cuadrante, para el cual igualmente usaremos un punto cercano a lo que se muestra en la gráfica como interseccion de las funciones.Por lo tanto ocuparemos el $x0=[-2;-1.8] $ obteniendo como el resultado el vector:
            $x*=[-2.4022;-1.6030]$ siendo el punto $x=-2.4022$, $y=-1.6030$

        \newpage    
        Pregunta $ 1b)$
        
            $ Recta(1) : x^2 + x -y^2=1 $
            
            $ Recta(2) : y-sen(x^2)=0  $
            
            
    
        \newpage
        
        \item En un sistema de tuberías (ver figura), los caudales de un cierto fluido en cada rama y las presiones en cada nodo de la red se relacionan mediante el siguiente sistema no lineal.
        
        \begin{align*} 
            p1 - p2 &= KQ^{1,75} \\ 
            p2 - p4 &= K_1 Q_1^{1,75}\\
            p2 - p3 &= K_2 Q_2^{1,75}
        \end{align*}
        
    \begin{figure}[H]
    \centering
    \includegraphics[width=7cm]{imagenEjec}
    \caption{Sistema de flujos}
    \end{figure}        
    
    Para un determinado fluido se sabe que
    \begin{align*}
    p_1&=75psi & p_3&=20psi &  p_4&=15psi & K&=2,35e^{-3} & K_1&=4,67e^{-3} & K_2&=3,72e^{-2}
    \end{align*}
    
    Usando el método de Newton en varias variables, determine los caudales en todas las ramas, y la presión en el nodo 2 suponiendo que $Q = Q_1 + Q_2 $. Use como valores iniciales: $ p_2^0= 50 psi,$  $Q_2^0=7$ y  $Q_1^0=16$
    
    
    Reescribiendo las ecuaciones con valores dados:
    
    \begin{align*} 
        75 - p2 &=  2.35e^{-3}(Q_1+Q_2)^{1.75} \\ 
        p2 - 15 &= 4.67e^{-3}Q_1^{1.75} \\
        p2 - 20 &= 4.72e^{-2}Q_2^{1.75}
    \end{align*}
    
    Escribimos vector de funciones de la forma $f(x) = 0$

   $$\begin{pmatrix}
        75 - p2 - 2.35e^{-3}(Q_1+Q_2)^{1.75} &= 0 \\
        \\
        p2 - 15 - 4.67e^{-3}Q_1^{1.75} &= 0  \\
        \\
        p2 - 20 - 4.72e^{-2}Q_2^{1.75} &= 0
    \end{pmatrix}$$
    
    Utilizando los valores iniciales Se llega al resultado de: 
    $$\begin{pmatrix}
        p_2 &=44.5717 \hspace{0.1cm} psi \\
        \\
        Q_1 &=15.9418 \\
        \\
        Q_2 &=8.0495
    \end{pmatrix}$$
    Por lo tanto como $Q = Q_1 + Q_2 $ implica que $ Q = 23.9913 $
    
    Al igual que el ejercicio 1b) de éste ítem, para lograr el objetivo de encontrar una raíz se limita a una cantidad de iteraciones iguales a 3, logrando así valores cercanos a la solución de la ecuación.
    
    Para realizar este ítem se utilizo el código de NewtonVariables.m
    \end{enumerate}
    \newpage

    
    \chapter{Conclusión}
      Del trabajo realizado se concluye lo siguiente, para el uso de ecuaciones no lineales, junto con esto los métodos que fueron aplicados para cada ítem, se logra  una amplia comprensión de como es el funcionamiento de cada uno de estos procesos para los casos a los cuales se aplicaron, esto quiere decir que se observó el comportamiento de la función en cada método, casos como el del punto fijo, secante y newton en donde la toma del punto inicial fue fundamental para ver si el método que aplicamos era el correcto, así se determinó que para este tipo de casos existía una convergencia local, ya que la dependencia de este punto era vital para que existiera convergencia. 
\\
En general, la aplicación de estos métodos se hicieron con el fin de aproximarse cada vez más a la solución, es por ello que se realizaban múltiples iteraciones para la mayoría de los casos y observando en todo momento su comportamiento. Finalmente, la realización del trabajo es fundamental para comprender de manera practica la teoría de los métodos de aproximación. 

\end{document}
