\documentclass{udpreport}
\title{Metodos Numéricos : Tarea 2}
\author{Andrés Comte, Benjamín Morales, Dagoberto Navarrete.}
\usepackage{amssymb}
\usepackage{amsmath}
\usepackage{graphicx}
\usepackage{float}
\usepackage{array}
\graphicspath{ {Imagenes/} }
\usepackage{listings}
\usepackage{color}




\begin{document}
\maketitle
\tableofcontents
\listoffigures
\chapter{Introducción}

\newpage

\chapter{Resolución de sistema de ecuaciones lineales}
	
 \section{Aplicación de los esquemas programados}
 \begin{enumerate}
 	\item   
 		\begin{enumerate}
 			\item 	Usando la norma $|| · ||_{\infty}$ calcule el numero condición de la matriz H, para n = 5 hasta n = 30.
Represente en una gráfica el resultado obtenido. Que puede decir de la matriz H.
 			
 			\item Sea $b_{i} = \sum_{j=1}^n a_{ij}$ para $i = 1...n$. Usando descomposición LU encuentre la solución $x_{a}$ del
sistema $H_{x} = b$ cuando $n = 5, 10, 15, 20$.
 				\begin{itemize} 				
 				\item Para n=5:
 				
 				$x_{a} = \left(\begin{array}{c} 1.0\\ 1.0\\ 1.0\\ 1.0\\ 1.0\\ 1.0\\ \end{array}\right)$
 				\\
 				\\
 				\item Para n=10:
 				
 				$x_{a} = \left(\begin{array}{c} 1.0\\ 1.0\\ 1.0\\ 1.0\\ 1.0\\ 1.0\\ 0.9998\\ 1.0\\ 0.9999\\ 1.0 \end{array}\right)$
 				\\
 				\\
 				\item Para n=15:
 				
 				$x_{a} = \left(\begin{array}{c} 1.0\\ 1.0\\ 0.9992\\ 1.008\\ 0.9865\\ 0.6493\\ 4.094\\ -11.3\\ 28.09\\ -30.78\\ 14.27\\ 5.74\\ 8.759\\ -22.04\\ 9.068\\ 0.1336\\ 29.05\\ -42.82\\ 26.48\\ -4.379 \end{array}\right)$
 				\\
 				\\
 				\newpage
 				\item Para n=20:
 				
 				$x_{a} = \left(\begin{array}{c} 1.0\\ 1.0\\ 1.0\\ 0.9929\\ 1.055\\ 0.7579\\ 1.695\\ -0.8953\\ 7.118\\ -14.6\\ 22.59\\ -5.381\\ -19.13\\ 19.89\\ 11.04\\ -26.6\\ 29.17\\ -27.26\\ 32.59\\ -26.64\\ 8.374\\ -8.23\\ 29.33\\ 9.701\\ -39.71\\ -9.034\\ 39.38\\ 2.024\\ -19.37\\ 8.151 \end{array}\right)$
 			\end{itemize}
 			\item Sea $x_{T} = (1, 1, . . . , 1)$ la solución exacta. Calcule las siguientes medidas $rfe = \frac{||x_{T}-x_{a}||_{\infty}}{||x_{T}||_{\infty}} , rbe = \frac{||b-Hx_{a}||_{\infty}}{||b||_{\infty}}$ y $\frac{rfe}{rbe}$ cuando n = 5, 10, 15, 20. Grafique ambos errores para cada n. Comente los
resultados.
 			\begin{figure}[H]
 				\centering
 				\includegraphics[width=9cm]{grafo1-rfe}
 				\caption{Gráfico del error relativo $rfe$}	
 			\end{figure}
 			
 			\begin{figure}[H]
 				\centering
 				\includegraphics[width=9cm]{grafo1-rbe}
 				\caption{Gráfico del error relativo $rfb$}		
 			\end{figure}
 			
 			\begin{figure}[H]
 				\centering
 				\includegraphics[width=9cm]{grafo1-div}
 				\caption{Gráfico de$\frac{rfe}{rbe}$}
	
 			\end{figure}
 		
 			En el primer gráfico se puede observar que para matrices de menor tamaño, el error es despreciable, pero a medida que el tamaño de la matriz aumenta también lo hace su error. Esto se produce debido al mal condicionamiento de la matriz de Hilbert.
 		\end{enumerate}
 		
 	Para resolver este problema se ocuparon los siguientes archivos: SolLU.m, FactorizacionLU.m, DiagUp.m y DiagDown.m	
 	\newpage
 		\item
 	 \begin{enumerate}
 	    \item Programe los tres algoritmos
 	    \item  Grafique la solución obtenida por cada método. Compare con la solución original:
 	    \item Mencione en una tabla, la cantidad de iteraciones y el tiempo de ejecución, que usaron cada uno de los métodos para obtener la solución aproximada. Comente sus resultados.
 	  \end{enumerate}
      
        \item
                \begin{enumerate}
            \item Suponiendo que m = n, encuentre la solución del sistema anterior vıa eliminación Gaussiana, cuando $n = 8, 15, 20, 50$.
            \begin{table}[H]
        \centering
            \begin{tabular} {|c|}
            \hline
            Valores de la solución para n=8 \\
            \hline
            x_{1}=  0,42005\\
            \hline
            x_{2}=  -1,86380\\
            \hline
            x_{3}=  0,06716\\
            \hline
            x_{4}=  8,71470\\
            \hline
            x_{5}=  -2,26160\\
            \hline
            x_{6}=  -13,872\\
            \hline
            x_{7}=  1,82990\\
            \hline
            x_{8}=  7,0035\\
            \hline
            \end{tabular}
        \end{table}
        
        \begin{table}[H]
        \centering
            \begin{tabular} { |c|}
            
            \hline
            Valores de la solución para n=15 \\
            \hline
            x_{1}= 0,66216\\
            \hline
            x_{2}=  -3,2558\\
            \hline
            x_{3}=  2,5351\\
            \hline
            x_{4}=  51,084\\
            \hline
            x_{5}=  -142,67\\
            \hline
            x_{6}=  -349,97\\
            \hline
            x_{7}=  1267,5\\
            \hline
            x_{8}=  1219,3\\
            \hline
            x_{9}=  -4939\\
            \hline
            x_{10}= -2212,5\\
            \hline
            x_{11}= 9501,1\\
            \hline
            x_{12}= 1972,1\\
            \hline
            x_{13}= -8758,3\\
            \hline
            x_{14}= -676,7\\
            \hline
            x_{15}= 3068,2\\
            \hline
            \end{tabular}
        \end{table}
        
        \begin{table}[H]
        \centering
            \begin{tabular} {|c|}
            
            \hline
            Valores de la solución para n=20 \\
            \hline
            x_{1}= 0,78893\\
            \hline
            x_{2}=  -3,242\\
            \hline
            x_{3}=  -4,634\\
            \hline
            x_{4}=  78,394\\
            \hline
            x_{5}=  -62,354\\
            \hline
            x_{6}=  -999,57\\
            \hline
            x_{7}=  1236,5\\
            \hline
            x_{8}=  7758,6\\
            \hline
            x_{9}=  -8888,8\\
            \hline
            x_{10}= -37882\\
            \hline
            x_{11}= 34023\\
            \hline
            x_{12}= 1,17E+05\\
            \hline
            x_{13}= -74840\\
            \hline
            x_{14}= -2,25E+05\\
            \hline
            x_{15}= 94271\\
            \hline
            x_{16}= 2,60E+05\\
            \hline
            x_{17}= -62902\\
            \hline
            x_{18}= -1,64E+05\\
            \hline
            x_{19}= 17167\\
            \hline
            x_{20}= 42838\\
            \hline
            \end{tabular}
        \end{table}

\centering  
\begin{tabular}{ |l|l| }
  \hline
  \multicolumn{2}{|c|}{Valores de la solución para n=50 } \\
  \hline
   x_{1}= 0,95399 & x_{26}= 1,16E+09\\
x_{2}= -2,0286 & x_{27}= 1,10E+09\\
x_{3}= -20,884 & x_{28}= -4,92E+08\\
x_{4}= 80,9 & x_{29}= -1,05E+09\\
x_{5}= 429,77 & x_{30}= 2,60E+08\\
x_{6}= -1853,9 & x_{31}= 5,09E+08\\
x_{7}= -7667,8 & x_{32}= -4,34E+08\\
x_{8}= 27248 & x_{33}= -2,05E+08\\
x_{9}= 1,04E+05 & x_{34}= 3,96E+08\\
x_{10}= -2,85E+05 & x_{35}= 2,03E+08\\
x_{11}= -1,09E+06 & x_{36}= -1,38E+08\\
x_{12}= 2,08E+06 & x_{37}= -5,73E+07\\
x_{13}= 8,92E+06 & x_{38}= 1,22E+08\\
x_{14}= -8,88E+06 & x_{39}= -1,72E+08\\
x_{15}= -5,28E+07 & x_{40}= -1,89E+08\\
x_{16}= 1,16E+07 & x_{41}= 1,51E+08\\
x_{17}= 2,09E+08 & x_{42}= 8,63E+06\\
x_{18}= 7,42E+07 & x_{43}= 5,75E+06\\
x_{19}= -5,27E+08 & x_{44}= 1,35E+08\\
x_{20}= -4,32E+08 & x_{45}= -8,43E+07\\
x_{21}= 7,91E+08 & x_{46}= -7,38E+07\\
x_{22}= 1,07E+09 & x_{47}= 7,20E+07\\
x_{23}= -5,20E+08 & x_{48}= 1,16E+07\\
x_{24}= -1,48E+09 & x_{49}= -2,34E+07\\
x_{25}= -3,58E+08 & $x_{50} = -2,30E+06$\\
\hline
\end{tabular}
\newpage
\item Grafique la función $f$ versus el polinomio $g$ para $n = 8, 15, 20, 50$. Comente sus resultados
    \begin{figure}[H]
    \centering
    \subfloat[$n=8$]{{\includegraphics[width=9cm]{datos-8}}}
    \subfloat[$n=15$]{{\includegraphics[width=9cm]{datos-15} }}
    \hfill
    \subfloat[$n=20$]{{\includegraphics[width=9cm]{datos-20} }}
    \subfloat[$n=50$]{{\includegraphics[width=9cm]{datos-50} }}
    \end{figure}
     
Los gráficos debiesen ser iguales, pero la gráfica del polinomio $g$ se dispara a valores muy altos cuando $t>0,5$

     \end{enumerate}

 	\item 
 	\begin{enumerate}
 			\item Aplique los métodos iterativos: Jacobi, Gauss-Seidel y Relajación con $w = 0.7$ y $w = 1.3$, con punto inicial $x(0) = (0, . . . , 0)$ para encontrar la distribución inicial $s^0$, cuando $n = 10, n = 20$ y $n = 40$.
 			\begin{itemize}
 				\item Al ocupar el método de Richardson, en ambos casos, la matriz definida por $I-M^5$ tiene un radio espectral mayor a 1. Siendo 1.0084 cuando $n=20$ y 1.0091 para $n=40$
 				\item Lo mismo sucede con el método de Jacobi. Cuando $n = 20$, el radio espectral de la matriz definida por $I-Q^{-1}M^5$ tiene un valor de 3.8417, mientras que vale 3.8517 cuando  $n=40$.
 				\item Al ocupar el método de Gauss-Seidel, se encuentre el mismo inconveniente. Con valores del radio espectral 1.0862 y 1.0945, respectivamente.
 				\item Por último, cuando se uitliza el método de Relajación, nuevamente, se encuentra el mismo problema. Esta vez, los valores del radio espectral cuando $n=20$ son 1.0222, 1.0338 y 1.0382, para $w=0.7, 1.3,1.6$ respectivamente. Pr otra parte, cuando $n=40$, este vale 1.0242, 1.0369, 1.0432 para los respectivos valores de $w$
 			\end{itemize}
 			\newpage
 			\item Para cada método, mencione el criterio de parada y tolerancia, la convergencia (usando el radio espectral), la cantidad de iteraciones, el tiempo usado.\\
 			Para todos los métodos se ocupó una tolerancia de 0.0001 y como criterio de parada se utilizó $\frac{\norm{x^{k+1}-x^{k}}_{2}}{\norm{x^{k+1}}_{2}}\leq 0.0001$. 
 			Como ninguno de los métodos convergió, no se puede calcular la cantidad de iteraciones ni el tiempo utilizado.
 			
		\end{enumerate} 	  

	\item
 		\begin{enumerate}
			\item Comprobar que el vector $x_{e} = \frac{1}{n+1}(1, 2, . . . , n)$  es la única solución del sistema anterior.\\
			Como la matriz de coeficientes no tiene filas nulas ni filas que sean multiplos de otras, se puede afirmar que el determinante siempre es distinto de 0. Esto nos demuestra que esta matriz tiene una solución única. Para demostrar que el vector $x_{e} = \frac{1}{n+1}(1,2,...,n)$ es solución del sistema, ocuparemos una matriz de $6 \times 6$ a modo de ejemplo.
			\begin{equation}
				\textrm{El sistema definido por} Ax = b
				\textrm{ Donde } A = \left(\begin{array}{cccccc} 2.0 & -1.0 & 0 & 0 & 0 & 0\\ -1.0 & 2.0 & -1.0 & 0 & 0 & 0\\ 0 & -1.0 & 2.0 & -1.0 & 0 & 0\\ 0 & 0 & -1.0 & 2.0 & -1.0 & 0\\ 0 & 0 & 0 & -1.0 & 2.0 & -1.0\\ 0 & 0 & 0 & 0 & -1.0 & 2.0 \end{array}\right)
				\textrm{ y } x_{e} = \left(\begin{array}{c} 0.1667\\ 0.3333\\ 0.5\\ 0.6667\\ 0.8333\\ 1.0 \end{array}\right)
			\end{equation}			 
			
			Luego 
			\begin{equation}
				b = Ax_{e}
			\end{equation}
			
			Entonces el vector b tiene un valor de $\left(\begin{array}{c} 0\\ 0\\ 1.11\cdot 10^{-16}\\ 0\\ -1.11\cdot 10^{-16}\\ 1.167 \end{array}\right)$
			Luego si calculamos el error relativo entre este vector $b$ y el vector $b$ real, definido por $\frac{\norm{b_{e}-b}}{\norm{b_{e}}}$ donde $b_{e}$ sería este último, obtenemos un valor de 0.1667. 
			
			Por lo tanto, se concluye que el vector $x_{e} = \frac{1}{n+1}$
			
			\item Usando la forma implementable de los métodos iterativos (Jacobi, Gauss-Seidel y Relajación con $w = 0.5, w = 1.6$) y la punto inicial $x^0 = (1, . . . , 1)$, encuentre la solución del sistema lineal cuando $n = 25, n = 50 $ y $ n = 60$.	
			\begin{itemize}
				\item Cuando se ocupa el método de Richardson, este no converge para ninguno de los tres casos, ya que los valores del radio espectral de la matriz definida por $I-A$ son mayores a 1. Teniendo valores de 2.9854, 2.9962, 2.9973 para $n=25, n=50$ y $n=60$ respectivamente.
				
				\item Al ocupar el método de Jacobi con una tolerancia de 0.0001, se obtienen los siguientes valores para:
				
					 n = 25: 
					$x_{a} = \left(\begin{array}{c} 0.03939\\ 0.07875\\ 0.1181\\ 0.1574\\ 0.1967\\ 0.2358\\ 0.275\\ 0.314\\ 0.353\\ 0.3917\\ 0.4305\\ 0.4691\\ 0.5077\\ 0.546\\ 0.5844\\ 0.6225\\ 0.6606\\ 0.6986\\ 0.7365\\ 0.7743\\ 0.8121\\ 0.8497\\ 0.8873\\ 0.9249\\ 0.9625 \end{array}\right) $	
					 n = 50:
					$x_{a} = \left(\begin{array}{c} 0.0215\\ 0.04301\\ 0.06448\\ 0.08595\\ 0.1074\\ 0.1288\\ 0.1501\\ 0.1715\\ 0.1927\\ 0.2139\\ 0.235\\ 0.2561\\ 0.277\\ 0.2979\\ 0.3187\\ 0.3394\\ 0.36\\ 0.3805\\ 0.4009\\ 0.4212\\ 0.4414\\ 0.4615\\ 0.4814\\ 0.5013\\ 0.521\\ 0.5406\\ 0.5601\\ 0.5795\\ 0.5987\\ 0.6179\\ 0.6369\\ 0.6558\\ 0.6746\\ 0.6934\\ 0.7119\\ 0.7305\\ 0.7489\\ 0.7672\\ 0.7854\\ 0.8036\\ 0.8217\\ 0.8398\\ 0.8577\\ 0.8756\\ 0.8935\\ 0.9113\\ 0.9291\\ 0.9468\\ 0.9646\\ 0.9823 \end{array}\right)$
										
					 n = 60:	
					$x_{a} = \left(\begin{array}{c} 0.0187\\ 0.03739\\ 0.05608\\ 0.07474\\ 0.09339\\ 0.112\\ 0.1306\\ 0.1491\\ 0.1676\\ 0.186\\ 0.2044\\ 0.2227\\ 0.2409\\ 0.2591\\ 0.2772\\ 0.2952\\ 0.3131\\ 0.3309\\ 0.3487\\ 0.3663\\ 0.3838\\ 0.4012\\ 0.4186\\ 0.4357\\ 0.4529\\ 0.4698\\ 0.4867\\ 0.5034\\ 0.5201\\ 0.5366\\ 0.553\\ 0.5692\\ 0.5854\\ 0.6014\\ 0.6174\\ 0.6332\\ 0.6489\\ 0.6644\\ 0.6799\\ 0.6953\\ 0.7106\\ 0.7257\\ 0.7408\\ 0.7557\\ 0.7706\\ 0.7853\\ 0.8001\\ 0.8147\\ 0.8293\\ 0.8437\\ 0.8581\\ 0.8725\\ 0.8868\\ 0.901\\ 0.9153\\ 0.9294\\ 0.9436\\ 0.9577\\ 0.9718\\ 0.9859 \end{array}\right)$
			
				\item Cuando se utiliza el método de Gauss-Seidel con la misma tolerancia, los resultados son los siguientes:
				
				n = 25:
				$x_{a} = \left(\begin{array}{c} 0.03939\\ 0.07875\\ 0.1181\\ 0.1574\\ 0.1967\\ 0.2358\\ 0.275\\ 0.314\\ 0.353\\ 0.3917\\ 0.4305\\ 0.4691\\ 0.5077\\ 0.546\\ 0.5844\\ 0.6225\\ 0.6606\\ 0.6986\\ 0.7365\\ 0.7743\\ 0.8121\\ 0.8497\\ 0.8873\\ 0.9249\\ 0.9625 \end{array}\right)$
				n = 50:
				$x_{a} = \left(\begin{array}{c} 0.021\\ 0.04198\\ 0.06295\\ 0.0839\\ 0.1048\\ 0.1257\\ 0.1466\\ 0.1674\\ 0.1882\\ 0.2089\\ 0.2296\\ 0.2502\\ 0.2707\\ 0.2912\\ 0.3116\\ 0.332\\ 0.3523\\ 0.3725\\ 0.3926\\ 0.4127\\ 0.4326\\ 0.4525\\ 0.4723\\ 0.4921\\ 0.5117\\ 0.5313\\ 0.5508\\ 0.5702\\ 0.5895\\ 0.6088\\ 0.6279\\ 0.647\\ 0.6661\\ 0.685\\ 0.7039\\ 0.7227\\ 0.7415\\ 0.7602\\ 0.7788\\ 0.7974\\ 0.816\\ 0.8345\\ 0.853\\ 0.8714\\ 0.8898\\ 0.9082\\ 0.9266\\ 0.945\\ 0.9633\\ 0.9817 \end{array}\right)$
				
				
				n = 60:
				$x_{a} = \left(\begin{array}{c} 0.0187\\ 0.03739\\ 0.05608\\ 0.07474\\ 0.09339\\ 0.112\\ 0.1306\\ 0.1491\\ 0.1676\\ 0.186\\ 0.2044\\ 0.2227\\ 0.2409\\ 0.2591\\ 0.2772\\ 0.2952\\ 0.3131\\ 0.3309\\ 0.3487\\ 0.3663\\ 0.3838\\ 0.4012\\ 0.4186\\ 0.4357\\ 0.4529\\ 0.4698\\ 0.4867\\ 0.5034\\ 0.5201\\ 0.5366\\ 0.553\\ 0.5692\\ 0.5854\\ 0.6014\\ 0.6174\\ 0.6332\\ 0.6489\\ 0.6644\\ 0.6799\\ 0.6953\\ 0.7106\\ 0.7257\\ 0.7408\\ 0.7557\\ 0.7706\\ 0.7853\\ 0.8001\\ 0.8147\\ 0.8293\\ 0.8437\\ 0.8581\\ 0.8725\\ 0.8868\\ 0.901\\ 0.9153\\ 0.9294\\ 0.9436\\ 0.9577\\ 0.9718\\ 0.9859 \end{array}\right)$
				\item Finalmente,al utilizar el método de Relajación, se se encuentran las siguientes soluciones, dependiendo del valor de $w$:
				\begin{itemize}
				\item $w = 0.5$:
				n = 25: $x_{a} = \left(\begin{array}{c} 0.04188\\ 0.08371\\ 0.1254\\ 0.167\\ 0.2083\\ 0.2495\\ 0.2903\\ 0.3308\\ 0.371\\ 0.4108\\ 0.4502\\ 0.4893\\ 0.5279\\ 0.5661\\ 0.6039\\ 0.6413\\ 0.6784\\ 0.7151\\ 0.7515\\ 0.7875\\ 0.8233\\ 0.8589\\ 0.8944\\ 0.9296\\ 0.9648 \end{array}\right)$ 
				n = 50: $x_{a} = \left(\begin{array}{c} 0.02703\\ 0.05402\\ 0.08095\\ 0.1078\\ 0.1345\\ 0.1611\\ 0.1875\\ 0.2137\\ 0.2397\\ 0.2654\\ 0.2908\\ 0.316\\ 0.3409\\ 0.3654\\ 0.3895\\ 0.4133\\ 0.4367\\ 0.4597\\ 0.4823\\ 0.5045\\ 0.5263\\ 0.5476\\ 0.5684\\ 0.5888\\ 0.6088\\ 0.6283\\ 0.6474\\ 0.666\\ 0.6842\\ 0.7019\\ 0.7192\\ 0.7361\\ 0.7526\\ 0.7687\\ 0.7845\\ 0.7998\\ 0.8148\\ 0.8295\\ 0.8438\\ 0.8579\\ 0.8717\\ 0.8852\\ 0.8986\\ 0.9117\\ 0.9246\\ 0.9374\\ 0.9501\\ 0.9627\\ 0.9751\\ 0.9876 \end{array}\right)$	
				
				n = 60: $x_{a} = \left(\begin{array}{c} 0.02591\\ 0.0518\\ 0.07762\\ 0.1034\\ 0.129\\ 0.1545\\ 0.1798\\ 0.2049\\ 0.2299\\ 0.2545\\ 0.279\\ 0.3031\\ 0.327\\ 0.3505\\ 0.3737\\ 0.3965\\ 0.4189\\ 0.441\\ 0.4627\\ 0.4839\\ 0.5047\\ 0.5251\\ 0.545\\ 0.5644\\ 0.5834\\ 0.6019\\ 0.62\\ 0.6376\\ 0.6546\\ 0.6713\\ 0.6874\\ 0.7031\\ 0.7183\\ 0.733\\ 0.7473\\ 0.7611\\ 0.7745\\ 0.7874\\ 0.7999\\ 0.812\\ 0.8237\\ 0.8351\\ 0.846\\ 0.8566\\ 0.8669\\ 0.8768\\ 0.8864\\ 0.8958\\ 0.9049\\ 0.9137\\ 0.9223\\ 0.9306\\ 0.9388\\ 0.9468\\ 0.9547\\ 0.9624\\ 0.9701\\ 0.9776\\ 0.9851\\ 0.9926 \end{array}\right)$	
				
				\item $w = 1.2$:
				n = 25: $x_{a} = \left(\begin{array}{c} 0.04028\\ 0.08052\\ 0.1207\\ 0.1608\\ 0.2008\\ 0.2406\\ 0.2803\\ 0.3199\\ 0.3592\\ 0.3984\\ 0.4373\\ 0.4761\\ 0.5146\\ 0.5529\\ 0.591\\ 0.6289\\ 0.6666\\ 0.7042\\ 0.7415\\ 0.7787\\ 0.8158\\ 0.8528\\ 0.8896\\ 0.9265\\ 0.9632 \end{array}\right)$
				n = 50: $x_{a} = \left(\begin{array}{c} 0.02336\\ 0.0467\\ 0.07\\ 0.09326\\ 0.1165\\ 0.1396\\ 0.1626\\ 0.1855\\ 0.2083\\ 0.231\\ 0.2536\\ 0.276\\ 0.2982\\ 0.3203\\ 0.3422\\ 0.3639\\ 0.3855\\ 0.4068\\ 0.4279\\ 0.4488\\ 0.4695\\ 0.49\\ 0.5102\\ 0.5302\\ 0.55\\ 0.5696\\ 0.5889\\ 0.608\\ 0.6269\\ 0.6456\\ 0.664\\ 0.6823\\ 0.7003\\ 0.7181\\ 0.7358\\ 0.7532\\ 0.7705\\ 0.7876\\ 0.8046\\ 0.8214\\ 0.8381\\ 0.8546\\ 0.8711\\ 0.8874\\ 0.9036\\ 0.9198\\ 0.9359\\ 0.952\\ 0.968\\ 0.984 \end{array}\right)$
				
				n = 60: $x_{a} = \left(\begin{array}{c} 0.02101\\ 0.04201\\ 0.06297\\ 0.0839\\ 0.1048\\ 0.1256\\ 0.1463\\ 0.1669\\ 0.1875\\ 0.2079\\ 0.2282\\ 0.2484\\ 0.2684\\ 0.2883\\ 0.308\\ 0.3276\\ 0.347\\ 0.3662\\ 0.3852\\ 0.404\\ 0.4225\\ 0.4409\\ 0.4591\\ 0.4771\\ 0.4948\\ 0.5123\\ 0.5295\\ 0.5466\\ 0.5634\\ 0.58\\ 0.5963\\ 0.6124\\ 0.6283\\ 0.6439\\ 0.6593\\ 0.6745\\ 0.6895\\ 0.7042\\ 0.7188\\ 0.7331\\ 0.7472\\ 0.7611\\ 0.7749\\ 0.7884\\ 0.8018\\ 0.815\\ 0.8281\\ 0.841\\ 0.8538\\ 0.8664\\ 0.879\\ 0.8914\\ 0.9037\\ 0.9159\\ 0.9281\\ 0.9401\\ 0.9522\\ 0.9642\\ 0.9761\\ 0.9881 \end{array}\right)$
				
				\item $w = 1.65$:	
				n = 25: $x_{a} = \left(\begin{array}{c} 0.03998\\ 0.07992\\ 0.1198\\ 0.1596\\ 0.1993\\ 0.239\\ 0.2784\\ 0.3178\\ 0.357\\ 0.396\\ 0.4349\\ 0.4736\\ 0.5121\\ 0.5504\\ 0.5886\\ 0.6266\\ 0.6644\\ 0.7021\\ 0.7396\\ 0.7771\\ 0.8144\\ 0.8516\\ 0.8888\\ 0.9259\\ 0.9629 \end{array}\right)$
				n = 50: $x_{a} = \left(\begin{array}{c} 0.02269\\ 0.04537\\ 0.06802\\ 0.09063\\ 0.1132\\ 0.1357\\ 0.1581\\ 0.1804\\ 0.2027\\ 0.2248\\ 0.2468\\ 0.2687\\ 0.2905\\ 0.3122\\ 0.3336\\ 0.355\\ 0.3761\\ 0.3972\\ 0.418\\ 0.4387\\ 0.4591\\ 0.4795\\ 0.4996\\ 0.5195\\ 0.5393\\ 0.5588\\ 0.5782\\ 0.5974\\ 0.6164\\ 0.6352\\ 0.6539\\ 0.6724\\ 0.6907\\ 0.7088\\ 0.7268\\ 0.7447\\ 0.7624\\ 0.7799\\ 0.7974\\ 0.8147\\ 0.8319\\ 0.849\\ 0.866\\ 0.8829\\ 0.8998\\ 0.9166\\ 0.9333\\ 0.95\\ 0.9667\\ 0.9833 \end{array}\right)$
				
				n = 60: $x_{a} =\left(\begin{array}{c} 0.02017\\ 0.04033\\ 0.06046\\ 0.08056\\ 0.1006\\ 0.1206\\ 0.1406\\ 0.1604\\ 0.1802\\ 0.1999\\ 0.2195\\ 0.239\\ 0.2583\\ 0.2776\\ 0.2967\\ 0.3157\\ 0.3345\\ 0.3532\\ 0.3717\\ 0.39\\ 0.4082\\ 0.4262\\ 0.4441\\ 0.4617\\ 0.4792\\ 0.4965\\ 0.5136\\ 0.5305\\ 0.5473\\ 0.5638\\ 0.5802\\ 0.5963\\ 0.6123\\ 0.628\\ 0.6436\\ 0.659\\ 0.6743\\ 0.6893\\ 0.7042\\ 0.7189\\ 0.7334\\ 0.7478\\ 0.762\\ 0.7761\\ 0.79\\ 0.8038\\ 0.8175\\ 0.8311\\ 0.8445\\ 0.8578\\ 0.8711\\ 0.8842\\ 0.8973\\ 0.9103\\ 0.9232\\ 0.9361\\ 0.9489\\ 0.9617\\ 0.9745\\ 0.9872 \end{array}\right)$
				\end{itemize}
			\end{itemize}
			\item Para cada metodo, mencione el criterio de parada y tolerancia, la convergencia (usando el radio espectral), la cantidad de iteraciones, el tiempo usado. Calcule $\frac{||x_{e}-x_{a}||_{\infty}}{||x_{e}||_{\infty}}$. Comente sus resultados.\\
			Para todos los métodos se ocupó una tolerancia de 0.0001 y como criterio de parada se utilizó $\frac{\norm{x^{k+1}-x^{k}}_{2}}{\norm{x^{k+1}}_{2}}\leq 0.0001$. 
			\begin{itemize}
				\item Jacobi:
					\begin{table}[H]
						\centering
						\begin{tabular}{|c|c|c|c|c|}
							\hline 
							n & Iteraciones & Tiempo (s) & Convergencia & Error \\
							\hline
							25 & 603 &0.2313 & 0.9927 & 0.0080 \\
							\hline
							50 & 1594 & 0.8390 & 0.9927 & 0.0314 \\
							\hline
							60 & 19999 & 1.1572 & 0.9987 & 0.0456 \\
							\hline
						\end{tabular}
					\end{table}
				\item Gauss-Seidel:
				 \begin{table}[H]
						\centering
						\begin{tabular}{|c|c|c|c|c|}
							\hline 
							n & Iteraciones & Tiempo & Convergencia & Error \\
							\hline
							25 &603 & 0.0339 & 0.9855 & 0.008 \\
							\hline
							50 & 1594 & 0.0716 & 0.9962 & 0.0314 \\
							\hline
							60 & 19999 & 0.1228 & 0.9973 & 0.0456 \\
							\hline
						\end{tabular}
					\end{table}
				\item Relajación:
					\begin{itemize}
					\item $w=0.5$
						\begin{table}[H]
						\centering
						\begin{tabular}{|c|c|c|c|c|}
							\hline 
							n & Iteraciones & Tiempo & Convergencia & Error \\
							\hline
							25 & 1065 & 0.0154 & 0.9971 & 0.0290 \\
							\hline
							50 & 2021 & 0.1959 & 0.9992 & 0.1210 \\
							\hline
							60 & 2372 & 0.0204 & 0.9995 & 0.1824\\
							\hline
						\end{tabular}
					\end{table}
					\item $w=1.2$
					\begin{table}[H]
						\centering
						\begin{tabular}{|c|c|c|c|c|}
							\hline 
							n & Iteraciones & Tiempo & Convergencia & Error \\
							\hline
							25 & 683 & 0.0126 & 0.9943 & 0.0152\\
							\hline
							50 & 1649 & 0.0251 & 0.9986 & 0.0610 \\
							\hline
							60 & 1972 & 0.0196 & 0.9990 & 0.0896  \\
							\hline
						\end{tabular}
					\end{table}
					\item $w=1.65$
					\begin{table}[H]
						\centering
						\begin{tabular}{|c|c|c|c|c|}
							\hline 
							n & Iteraciones & Tiempo & Convergencia & Error \\
							\hline
							25 & 594 & 0.0123 & 0.9934 & 0.0126 \\
							\hline
							50 & 1481 & 0.0176 & 0.9983 & 0.0500 \\
							\hline
							60 & 1802 & 0.0180 & 0.9988 & 0.0732 \\
							\hline
						\end{tabular}
					\end{table}
					\end{itemize}			
			\end{itemize}
 	        Se puede observar que la cantidad de iteraciones y el error de los métodos de Gauss-Seidel y Jacobi tienen los mismos valores, aunque este último toma más tiempo.
 	        \\
			Por otra, parte al analizar los resultados del método de Relajación para los diferentes valores de $w$, se puede percatar de que a entre más grande es el valor de $w$, más rápido converge, además de que disminuye el error. \newpage
 	        \item Si en lugar de usar la forma implementable del método de Gauss-Seidel, se usa la forma general $x^{k+1} = (I-Q^{-I}_{G-S}A)x^k+Q^{-I}_{G-S}b$, cual es la solución del sistema y cual es el tiempo requerido para $n = 25$ y $n = 50$? Compare la solución obtenida en (b) con la que encuentra mediante este proceso.\\
 	        Los resultados encontrados al utilizar la forma general del método de Gauss-Seidel, son:
 	        
 	        n = 25: 
 	       	$x_{a} = \left(\begin{array}{c} 0.03916\\ 0.0783\\ 0.1174\\ 0.1565\\ 0.1955\\ 0.2345\\ 0.2734\\ 0.3122\\ 0.351\\ 0.3897\\ 0.4283\\ 0.4669\\ 0.5053\\ 0.5437\\ 0.582\\ 0.6203\\ 0.6584\\ 0.6965\\ 0.7346\\ 0.7726\\ 0.8105\\ 0.8485\\ 0.8864\\ 0.9243\\ 0.9621 \end{array}\right) $
 	        n = 50:
 	        $x_{a} = \left(\begin{array}{c} 0.03916\\ 0.0783\\ 0.1174\\ 0.1565\\ 0.1955\\ 0.2345\\ 0.2734\\ 0.3122\\ 0.351\\ 0.3897\\ 0.4283\\ 0.4669\\ 0.5053\\ 0.5437\\ 0.582\\ 0.6203\\ 0.6584\\ 0.6965\\ 0.7346\\ 0.7726\\ 0.8105\\ 0.8485\\ 0.8864\\ 0.9243\\ 0.9621 \end{array}\right)$
 	        
 	        Con esta información se pueden armar las siguientes tabla comparativas:
 	        \begin{itemize}
 	        	\item Para n=25: \begin{table}[H]
 	        			\centering
 	        			\begin{tabular}{|c|c|c|}
 	        				\hline 
							 & General & Implementable \\
							 \hline
							 Tiempo & 0.0183 & 0.0339 \\
							 \hline
							 Iteraciones & 320 & 603\\
							 \hline
							 Error & 0.0055 & 0.0080\\
							\hline
 	        			\end{tabular}
 	        		\end{table}
 	        		\item Para n=50: \begin{table}[H]
 	        			\centering
 	        			\begin{tabular}{|c|c|c|}
 	        				\hline 
							 & General & Implementable \\
							 \hline
							 Tiempo & 0.0167 & 0.0716 \\
							 \hline
							 Iteraciones & 879 & 1594 \\
							 \hline
							 Error & 0.0220 & 0.0314\\
							\hline
 	        			\end{tabular}
 	        		\end{table}
 	        		Con estos datos, se puede concluír que la forma general del método de Gauss-Seidel es mucho más efectiva a la hora de resolver un sistema que su contraparte implementable. Esto debido a que en ambos casos, la forma general presenta un menor tiempo de ejecución, una menor cantidad de iteracionas y un menor error.
 	        		\\
 	        		Para responder esta pregunta se utilizaron los archivos GaussSeidel.m, Jacobi2.m, gs-implementable.m y Richardson.m
 	        \end{itemize}
 	 \end{enumerate}
 \end{enumerate}
 

\newpage
\chapter{Aproximación de funciones} 
\begin{enumerate}
    
\vspace{0.9cm}

Se desea dibujar el contorno del dibujo de Snoopy:


\begin{figure}[H]
    \centering
    \includegraphics[width=10cm]{enun1}
    \caption{enunciado.} \label{fig:enun1}
\end{figure}




\item Utilizando la interpolación por Spline cubico, obtenga las dos curvas que aproximan el contorno del  Snoopy.


\begin{figure}[H]
    \centering
    \includegraphics[width=9cm]{interp_spline}
    \caption{interpolación Spline cubica.} \label{fig:interp_spline}
\end{figure}

\begin{figure}[H]
    \centering
    \includegraphics[width=9cm]{interp_spline_csaps}
    \caption{interpolación Spline cubica suavizante.} \label{fig:interp_spline_csaps}
\end{figure}
\newpage
La función utilizada es:
 
\begin{itemize}
\item interp
\end{itemize}
obs: en esta función se puede cambiar el comando ya sea por spline,csaps u otras, así se ve su comportamiento en cada caso.
\item Como podría utilizar el polinomio de interpolación de Lagrange para aproximar la parte inferior del contorno del Snoopy? Trace el grafico que obtiene en esta aproximación.

Primero, para plantear el uso del polinomio de interpolación de Lagrange se hará uso de la definición de este polinomio interporlante la cual se planteará a continuación.

 \begin{equation}
 p(x)=\sum_{i=0}^n yi*Li(x)
\end{equation}
\\ De lo anterior, se define inmediatamente el valor de Li de la siguiente manera:
 \begin{equation}
 Lk(x)=\prod_{j=0,j\not=k}^{n}
\end{equation}
Luego, despues de definir como se utilizará el polinimio interpolante de Lagrange, se aplica el algoritmo en MATLAB con el fin de obtener graficamente lo que se plantea. 
\begin{figure}[H]
    \centering
    \includegraphics[width=9cm]{int_lagrange_inf}
    \caption{interpolación Lagrange parte inferior del Snoopy.} \label{fig:int_lagrange_inf}
\end{figure}
Usando la misma interpolación polinómica de Lagrange y bajo el mismo algoritmo se grafica igualmente la parte superior del auto en cuestión.
\begin{figure}[H]
    \centering
    \includegraphics[width=9cm]{int_lagrange_aprox_parte_sup}
    \caption{interpolación lagrange parte superior del Snoopy.} \label{fig:int_lagrange_aprox_parte_sup}
\end{figure}
La función utilizada es:
 
\begin{itemize}
\item lagrang(función que realiza el polinimio int)
\item lagrange(script que pide los datos)
\end{itemize}
\newpage


  
 \chapter{Integración de funciones} 

 \begin{enumerate}
 \item Use la fórmula del trapecio compuesta (para 6 subintervalos) para calcular la fuerza total ejercida por el agua en la cara de la presa.
 \\ 
 \\
 Usando la fórmula del trapecio compuesto se obtiene que la fuerza total ejercida por el agua en la cara de la presa es la siguiente:
 \\
 \\
 f=1.0854e+10
 resultó ser la siguiente imagen.
 \begin{figure}[H]
    \centering
    \includegraphics[width=9cm]{fuerza_trap}
    \caption{fuerza o presión en cada tramo calculado.} \label{fig:fuerza_trap}
\end{figure}
 \begin{figure}[H]
    \centering
    \includegraphics[width=9cm]{presionfin}
    \caption{presión en cota a mayor profundidad.} \label{fig:presionfin}
\end{figure}

 
\begin{itemize}
\item trapecio
\end{itemize}

 \vspace{0.8cm}
 
 \item Usando interpolación de Lagrange, dibuje la curva del lado derecho de la  figura.
 
  \begin{figure}[H]
    \centering
    \includegraphics[width=9cm]{curva_derecha_lagrange}
    \caption{Curva lado derecho de la presa, interpolación polinómica de Lagrange.} \label{fig:curva_derecha_lagrange}
  
\end{figure}


 \item Usando interpolación por spline, dibuje la curva del lado izquierdo de la  figura.
 \\
 \begin{figure}[H]
    \centering
    \includegraphics[width=9cm]{curvaizq_spline}
    \caption{Curva lado izquierdo de la presa, interpolación por spline cubico.} \label{fig:curvaizq_spline}
\end{figure}

 \end{enumerate}
 \end{enumerate}

\newpage

    
    
\chapter{Conclusión}



\end{document}
