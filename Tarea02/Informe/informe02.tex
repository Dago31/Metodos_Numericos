\documentclass{udpreport}
\title{Metodos Numéricos : Tarea 2}
\author{Andrés Comte, Benjamín Morales, Dagoberto Navarrete.}
\usepackage{amssymb}
\usepackage{amsmath}
\usepackage{graphicx}
\usepackage{float}
\usepackage{array}
\graphicspath{ {Imagenes/} }
\usepackage{listings}
\usepackage{color}




\begin{document}
\maketitle
\tableofcontents
\listoffigures
\chapter{Introducción}

En el presente trabajo, el objetivo principal fue analizar algoritmos que permitieran resolver diversos problemas matemáticos, en el caso de resolución de sistemas de ecuaciones lineales se utilizaron algoritmos tales como eliminación Gaussiana, sustitución hacia atrás, factorización LU, factorización Cholesky y métodos iterativos tales como Jacobi, Gauss-Seidel y Relajación, para la aproximación e integración de funciones se utilizaron algoritmos de interpolación de Spline cúbico e  interpolación de Lagrange, en el caso de la integración de funciones se ocupó también el algoritmo del trapecio compuesto.

\newpage

\chapter{Resolución de sistema de ecuaciones lineales}
	
 \section{Aplicación de los esquemas programados}
 \begin{enumerate}
 	\item   
 		\begin{enumerate}
 			\item 	Usando la norma $|| · ||_{\infty}$ calcule el numero condición de la matriz H, para n = 5 hasta n = 30.
Represente en una gráfica el resultado obtenido. Que puede decir de la matriz H.
\begin{figure}[H]
 				\centering
 				\includegraphics[width=9cm]{Grafico1a.png}
 				\caption{Gráfico numero condición v/s $N$ }	
 			\end{figure}
 			
 			\item Sea $b_{i} = \sum_{j=1}^n a_{ij}$ para $i = 1...n$. Usando descomposición LU encuentre la solución $x_{a}$ del
sistema $H_{x} = b$ cuando $n = 5, 10, 15, 20$.
 				\begin{itemize} 				
 				\item Para n=5:
 				
 				$x_{a} = \left(\begin{array}{c} 1.0000\\ 1.0000\\ 1.0000\\ 1.0000\\ 1.0000\end{array}\right)$
 				\\
 				\\
 				\item Para n=10:
 				
 				$x_{a} = \left(\begin{array}{c} 1.0000\\ 1.0000\\1.0000\\1.0000\\ 1.0000\\ 1.0001\\ 0.9998\\ 1.0002\\ 0.9999\\1.0000 \end{array}\right)$
 				\\
 				\\
 				\item Para n=15:
 				
 				$x_{a} = \left(\begin{array}{c} 1.0000\\ 1.0000\\ 0.9998\\ 1.0024\\ 0.9879\\1.0004\\ 1.3100\\ -0.7814\\  6.4136\\ -9.4048\\ 14.2587\\ -10.2060\\  7.0491\\  -0.8897\\  1.26008 \end{array}\right)$
 				\\
 				\\
 
 				\item Para n=20:
 				
 				$x_{a} = \left(\begin{array}{c} 1.0000\\ 1.0000\\ 0.9992\\ 1.0082\\ 0.9865\\ 0.6493\\ 4.0935\\-11.3012\\ 28.0860\\  -30.7814\\14.2669\\  5.7401\\  8.7586\\-22.0413\\ 9.0679\\  0.1336\\ 29.0535\\ -42.8189\\  26.4788\\ - -4.3793\end{array}\right)$
 			\end{itemize}
 			\item Sea $x_{T} = (1, 1, . . . , 1)$ la solución exacta. Calcule las siguientes medidas $rfe = \frac{||x_{T}-x_{a}||_{\infty}}{||x_{T}||_{\infty}} , rbe = \frac{||b-Hx_{a}||_{\infty}}{||b||_{\infty}}$ y $\frac{rfe}{rbe}$ cuando n = 5, 10, 15, 20. Grafique ambos errores para cada n. Comente los
resultados.
 			\begin{figure}[H]
 				\centering
 				\includegraphics[width=9cm]{rfe.png}
 				\caption{Gráfico del error relativo $rfe$}	
 			\end{figure}
 			
 			\begin{figure}[H]
 				\centering
 				\includegraphics[width=9cm]{rbe.png}
 				\caption{Gráfico del error relativo $rfb$}		
 			\end{figure}
 			
 			\begin{figure}[H]
 				\centering
 				\includegraphics[width=9cm]{rferbe.png}
 				\caption{Gráfico de$\frac{rfe}{rbe}$}
	
 			\end{figure}
 		
 			
 			En lo gráficos se observa como va aumentando el error a  medida que va aumentando n, esto se debe a que la matriz de Hilbert esta mal condicionada, por que al menor cambio de uno de sus valores aumenta el error.
 		\end{enumerate}
 		
 	
 	\newpage
 		\item
 	 \begin{enumerate}
 	 
 	    \item Programe los tres algoritmos
 	    \item  Grafique la solución obtenida por cada método. Compare con la solución original:
 	    \item Mencione en una tabla, la cantidad de iteraciones y el tiempo de ejecución, que usaron cada uno de los métodos para obtener la solución aproximada. Comente sus resultados.
 	  \end{enumerate}
      
        \item
                \begin{enumerate}
            \item Suponiendo que m = n, encuentre la solución del sistema anterior vıa eliminación Gaussiana, cuando $n = 8, 15, 20, 50$.
            \begin{table}[H]
        \centering
            \begin{tabular} {|c|}
            \hline
            Valores de la solución para n=8 \\
            \hline
            $x_{1}=  0.7526$\\
            \hline
            $x_{2}=   0.0000$\\
            \hline
            $x_{3}=  -4.61664$\\
            \hline
            $x_{4}=  0.0000$\\
            \hline
            $x_{5}=  9.0760$\\
            \hline
            $x_{6}=   0.0000$\\
            \hline
            $x_{7}=  -5.1736$\\
            \hline
            $x_{8}=  0.0000$\\
            \hline
            \end{tabular}
        \end{table}
        
        \begin{table}[H]
        \centering
            \begin{tabular} { |c|}
            
            \hline
            Valores de la solución para n=15 \\
            \hline
            $x_{1}= 0.0001e+04$\\
            \hline
            $x_{2}=  0.0000$\\
            \hline
            $x_{3}=  -0.0021e+04$\\
            \hline
            $x_{4}=  0.0000$\\
            \hline
            $x_{5}=  0.0278e+04$\\
            \hline
            $x_{6}=  0.0000$\\
            \hline
            $x_{7}=  -0.1871e+04$\\
            \hline
            $x_{8}=   0.0000$\\
            \hline
            $x_{9}=  0.6582e+04$\\
            \hline
            $x_{10}= 0.0000$\\
            \hline
            $x_{11}= -1.2073e+04$\\
            \hline
            $x_{12}= 0.0000$\\
            \hline
            $x_{13}= 1.0850e+04$\\
            \hline
            $x_{14}= 0.0000$\\
            \hline
            $x_{15}= -0.3745e+04$\\
            \hline
            \end{tabular}
        \end{table}
        
        \begin{table}[H]
        \centering
            \begin{tabular} {|c|}
            
            \hline
            Valores de la solución para n=20 \\
            \hline
            $x_{1}= 0.0000$\\
            \hline
            $x_{2}=  0.0000$\\
            \hline
            $x_{3}=   -0.0002e+05$\\
            \hline
            $x_{4}=  0.0000$\\
            \hline
            $x_{5}=  0.0033e+05$\\
            \hline
            $x_{6}=  0.0000$\\
            \hline
            $x_{7}=  0.0306e+05$\\
            \hline
            $x_{8}=  0.0000$\\
            \hline
            $x_{9}=  0.1717e+05$\\
            \hline
            $x_{10}= 0.0000$\\
            \hline
            $x_{11}=  -0.5858e+05$\\
            \hline
            $x_{12}= 0.0000$\\
            \hline
            $x_{13}= 1.2102e+05$\\
            \hline
            $x_{14}= 0.0000$\\
            \hline
            $x_{15}= -1.4679e+05$\\
            \hline
            $x_{16}= 0.0000$\\
            \hline
            $x_{17}= 0.9560e+05$\\
            \hline
            $x_{18}= 0.0000$\\
            \hline
            $x_{19}= -0.2567e+05$\\
            \hline
            $x_{20}= 0.0000$\\
            \hline
            \end{tabular}
        \end{table}

\centering  
\begin{tabular}{ |l|l| }
  \hline
  \multicolumn{2}{|c|}{Valores de la solución para n=50 } \\
  \hline
   $x_{1}=0.0000$ & $x_{26}= -0.8658e+09$\\
$x_{2}= 0.0000$ & $x_{27}=  2.3850e+09$\\
$x_{3}= 0.0000$ & $x_{28}= -0.1702e+09$\\
$x_{4}= 0.0000$ & $x_{29}= -1.5634e+09$\\
$x_{5}= 0.0000$ & $x_{30}= 0.1351e+09$\\
$x_{6}= 0.0000$ & $x_{31}= 0.6060e+09$\\
$x_{7}= -0.0001e+09$ & $x_{32}= 0.3682e+09$\\
$x_{8}=  -0.0003e+09$ & $x_{33}= -0.0247e+09$\\
$x_{9}= 0.0007e+09$ & $x_{34}=  -0.3568e+09$\\
$x_{10}= 0.0040e+09$ & $x_{35}= 0.1459e+09$\\
$x_{11}= 0.0001e+09$ & $x_{36}=  0.1836e+09$\\
$x_{12}= -0.0263e+09$ & $x_{37}=  -0.1136e+09$\\
$x_{13}= -0.0296e+09$ & $x_{38}= -0.0344e+09$\\
$x_{14}=  0.0949e+09$ & $x_{39}= -0.0684e+09$\\
$x_{15}=  0.1986e+09$ & $x_{40}= -0.0061e+09$\\
$x_{16}=  -0.1868e+09$ & $x_{41}= 0.0185e+09$\\
$x_{17}= -0.7121e+09$ & $x_{42}= -0.0967e+09$\\
$x_{18}= 0.1293e+09$ & $x_{43}=  -0.0026e+09$\\
$x_{19}= 1.6984e+09$ & $x_{44}= 0.0503e+09$\\
$x_{20}= 0.3302e+09$ & $x_{45}= -0.0072e+09$\\
$x_{21}=-2.9016e+09$ & $x_{46}= 0.0180e+09$\\
$x_{22}= -1.1464e+09$ & $x_{47}= 0.0108e+09$\\
$x_{23}=  3.5872e+09$ & $x_{48}= -0.0022e+09$\\
$x_{24}= 1.5787e+09$ & $x_{49}= -0.0006e+09$\\
$x_{25}=  -3.2413e+09$ & $x_{50} = -0.0005e+09$
\\
\hline
\end{tabular}
\newpage
\item Grafique la función $f$ versus el polinomio $g$ para $n = 8, 15, 20, 50$. Comente sus resultados
    \begin{figure}[H]
    \centering
    \subfloat[$n=8$]{{\includegraphics[width=9cm]{n8.png}}}
    \subfloat[$n=15$]{{\includegraphics[width=9cm]{n15.png} }}
    \hfill
    \subfloat[$n=20$]{{\includegraphics[width=9cm]{n20.png} }}
    \subfloat[$n=50$]{{\includegraphics[width=9cm]{n40.png} }}
    \end{figure}
    Para valores muy grandes de $n$ se observa como $g(ti)$ se va a alejando de la gráfica de $f(ti)$.

     \end{enumerate}

 	\item 
 	\begin{enumerate}
 			\item Aplique los métodos iterativos: Jacobi, Gauss-Seidel y Relajación con $w = 0.7$ y $w = 1.3$, con punto inicial $x(0) = (0, . . . , 0)$ para encontrar la distribución inicial $s^0$, cuando $n = 10, n = 20$ y $n = 40$.
 			\begin{itemize}
 				\item Al utilizar el método de Jacobi,el radio espectral para todos los $n$ nos arrojo un valor mayor a 1 por lo que no se pudo llegar a ninguna solución con este método.
 				\item En caso Del método de Gauss Seidel este divergió para los valores de $n = 20$ y $n = 40$, ya que nos arrojo un radio espectral de valor  1.000000000000001, pero en el caso de $n = 10$ converge ya que nos arroja un radio espectral de 0.999999999765406, por lo tanto la solución es:\\
 				
 				$s^{0} = \left(\begin{array}{c} 0.870774591784717\\ -1.287107307070359\\ 0.427396520772320\\ 0.357447273512588\\ -0.410646853699434\\ 0.128132826062855\\ 0.036089036182169\\ -0.056888992915167\\ 0.042348308511476\\-0.015216418759887
 				\end{array}\right)*10^4$
 				\\
 				\\
 				
 				\item Finalmente al aplicar el método de Relajacion, este diverge para los valores de $n= 20$ y $n = 40$, para ambos valores de $w$ ya que arroja un radio espectral de 1.000000000000002, en cambio para el valor de $n = 10$ converge ya que para ambos valores de $w$, $0.7$ y $1.3$, da un radio espectral de 0.999999999939976 y 0.999999999908685 respectivamente, por lo que las soluciones son:
 				\begin{itemize}
 				    \item para $ w = 0.7$\\
 				    $s^{0} = \left(\begin{array}{c} 3.222238945010367\\ -4.071280113608322\\ 0.335654485404636\\ 1.624918680817723\\ -0.573075696236457\\ -0.321910073468050\\ -0.014992075512375\\  0.209509521028656\\  0.085233972063552\\-0.105383007515083
 				\end{array}\right)*10^3$
 				\\
 				\\
 				\item para $ w = 1.3$\\
 				    $s^{0} = \left(\begin{array}{c} 4.411341632114369\\ -5.842040650271944\\ 0.922131164431082\\ 2.156010133600716\\ -1.175434495347176\\ -0.175757643009034\\  0.096574848279233\\  0.127729302443524\\  0.088953874115745\\-0.097355303679830
 				\end{array}\right)*10^3$
 				\\
 				\\
 				    
 				\end{itemize}
 			\end{itemize}
 		\item Para cada método, mencione el criterio de parada y tolerancia, la convergencia (usando el radio espectral), la cantidad de iteraciones, el tiempo usado.\\
 			Para todos los métodos se ocupó una tolerancia de 0.00001 y como criterio de parada se utilizó $\frac{\norm{x^{k+1}-x^{k}}_{2}}{\norm{x^{k+1}}_{2}}\leq 0.00001$. \\
 			\begin{itemize}
				\item Jacobi:
					\begin{table}[H]
						\centering
						\begin{tabular}{|c|c|c|c|}
							\hline 
							n & Iteraciones & Tiempo (s) & Convergencia  \\
							\hline
							10 & 0 & 0 & 4.463797819674746  \\
							\hline
							20 & 0 & 0 & 7.360445579609070 \\
							\hline
							40 & 0 & 0 & 11.429028652113038  \\
							\hline
						\end{tabular}
					\end{table}
				\item Gauss-Seidel:
				 \begin{table}[H]
						\centering
						\begin{tabular}{|c|c|c|c|}
							\hline 
							n & Iteraciones & Tiempo & Convergencia  \\
							\hline
							10 & 85895 & 0.353124 & 0.999999999765406  \\
							\hline
							20 & 0 & 0 & 1.000000000000001  \\
							\hline
							40 & 0 & 0 &  1.000000000000001  \\
							\hline
						\end{tabular}
					\end{table}
				\item Relajación:
					\begin{itemize}
					\item $w=0.7$
						\begin{table}[H]
						\centering
						\begin{tabular}{|c|c|c|c|c|}
							\hline 
							n & Iteraciones & Tiempo & Convergencia  \\
							\hline
							10 & 84278 & 0.360452 & 0.999999999939976  \\
							\hline
							20 & 0 & 0 & 1.000000000000000  \\
							\hline
							40 & 0 & 0 & 1.000000000000001 \\
							\hline
						\end{tabular}
					\end{table}
					\item $w=1.3$
					\begin{table}[H]
						\centering
						\begin{tabular}{|c|c|c|c|c|}
							\hline 
							n & Iteraciones & Tiempo & Convergencia  \\
							\hline
							10 & 83804 & 0.329130 & 0.999999999908685 \\
							\hline
							20 & 0 & 0 & 1.000000000000002  \\
							\hline
							40 & 0 & 0 & 1.000000000000001  \\
							\hline
						\end{tabular}
					\end{table}
					\end{itemize}			
			\end{itemize}
 			
 			
 			
		\end{enumerate} 	  

	\item 
 		\begin{enumerate}
			\item Comprobar que el vector $x_{e} = \frac{1}{n+1}(1, 2, . . . , n)$  es la única solución del sistema anterior.\\\par 
			Se busco la solucion del sistema utilizando el metodo de Gauss-Jordan, escalonando la matriz.
			
			%\begin{equation}
				\left[ \begin{array}{cccccc}
                1  & \frac{-1}{2}    & 0      &        &        &\\
                0 &  \frac{3}{2}    & -1     &        &        &\\
                0  &  -1   & 2      & -1     &        &     \\
                &\ddots & \ddots & \ddots & \ddots & \\
                &       &        & -1     & 2      & -1 \\
                &       &        &      & -1      & 2 \\
                
                \end{array}\right|  \left{ \begin{array}{c}
                0\\
                0\\
                0\\
                0\\
                \vdots \\
                1\\
                \end{array}\right] \rightarrow 
                \left[ \begin{array}{cccccc}
                1  & \frac{-1}{2}    & 0      &        &        &\\
                0  &  1    & \frac{-2}{3}    &        &        &\\
                0  &  0    & 1      & \frac{-3}{4}   &        &     \\
                &\ddots & \ddots & \ddots & \ddots & \\
                &       &        & -1     & 2      & -1 \\
                &       &        &      & -1      & 2 \\
                \end{array}\right| \left{ \begin{array}{c}
                0\\
                0\\
                0\\
                0\\
                \vdots\\
                1\\
                \end{array}\right] \rightarrow
                
                \left[ \begin{array}{cccccc}
                1  & \frac{-1}{2}  & 0             &              &        &\\
                0  &  1            & \frac{-2}{3}  &              &        &\\
                0  &  0            & 1             & \frac{-3}{4} &        & \\
                &\ddots         & \ddots        & \ddots       & \ddots & \\
                &               &               & 0            & 1     & \frac{-n}{n+1} \\
                 &       &        &      & 0     & \frac{n}{n+1} \\
                \end{array}\right| \left{ \begin{array}{c}
                0\\
                0\\
                0\\
                0\\
                \vdots\\
                1\\
                \end{array}\right] \rightarrow
                \next
                \left[ \begin{array}{cccccc}
                1  & 0     & 0      &        &        &\\
                0  &  1    & 0      &        &        &\\
                0  &  0    & 1      & 0      &        &     \\
                &\ddots & \ddots & \ddots & \ddots & \\
                &       &        & 0      & 1      & 0 \\
                &       &        &      & 0      & 1 \\
                \end{array}\right| \left{ \begin{array}{c}
                \frac{1}{5}\\
                \frac{2}{5}\\
                \frac{3}{5}\\
                \vdots \\
                \frac{n-1}{n+1} \\
                \frac{n}{n+1}\\
                \end{array} \right]
                \par
                
			%\end{equation}			 
		 
			Por lo tanto, se concluye que la solucion desl sistema anterios  $x_{e} = \frac{1}{n+1}(1,2,3,...,n)$\par
			
			\item Usando la forma implementable de los métodos iterativos (Jacobi, Gauss-Seidel y Relajación con $w = 0.5, w = 1.6$) y la punto inicial $x^0 = (1, . . . , 1)$, encuentre la solución del sistema lineal cuando $n = 25, n = 50 $ y $ n = 60$.	
			\begin{itemize}
				
				\item Al ocupar el método de Jacobi con una tolerancia de 0.000001, se obtienen los siguientes valores para:
				
					 n = 25: 
					$x_{a} = \left(\begin{array}{c} 0.0385\\ 0.0769\\ 0.1154\\ 0.1539\\ 0.1924\\ 0.2308\\ 0.2693\\ 0.3078\\ 0.3464\\ 0.3847\\ 0.4232\\ 0.4616\\ 0.5001\\ 0.5385\\ 0.5770\\ 0.6155\\ 0.6539\\ 0.6924\\ 0.7308\\ 0.7693\\ 0.8077\\ 0.8464\\ 0.8846\\ 0.9231\\ 0.9615 \end{array}\right) $	
					 n = 50:
					$x_{a} = \left(\begin{array}{c} 0.0196\\ 0.0393\\ 0.0589\\ 0.0785\\ 0.0981\\ 0.1178\\ 0.1374\\ 0.1570\\ 0.1766\\ 0.1963\\ 0.2159\\ 0.2355\\ 0.2551\\ 0.2747\\ 0.2944\\ 0.3140\\ 0.3336\\ 0.3532\\ 0.3728\\ 0.3924\\ 0.4121\\ 0.4317\\ 0.4513\\ 0.4709\\ 0.4905\\ 0.5101\\ 0.5297\\ 0.5493\\ 0.5689\\ 0.5885\\ 0.6081\\ 0.6277\\ 0.6473\\ 0.6669\\ 0.6865\\ 0.7061\\ 0.7257\\ 0.7453\\ 0.7649\\ 0.7845\\ 0.8041\\ 0.8237\\ 0.8433\\ 0.8629\\ 0.8825\\ 0.9021\\ 0.9216\\ 0.9412\\ 0.9608\\ 0.9804 \end{array}\right)$
										
					 n = 60:	
					$x_{a} = \left(\begin{array}{c} 0.0164\\ 0.0328\\ 0.0492\\ 0.0657\\ 0.0821\\ 0.0985\\ 0.1149\\ 0.1313\\ 0.1477\\ 0.1641\\ 0.1806\\ 0.1970\\ 0.2134\\ 0.2298\\ 0.2462\\ 0.2626\\ 0.2790\\ 0.2954\\ 0.3118\\ 0.3282\\ 0.3446\\ 0.3610\\ 0.3774\\ 0.3938\\ 0.4102\\ 0.4266\\ 0.4430\\ 0.4594\\ 0.4758\\ 0.4922\\ 0.5086\\ 0.5250\\ 0.5414\\ 0.5578\\ 0.5742\\ 0.5906\\ 0.6070\\ 0.6233\\ 0.6397\\ 0.6561\\ 0.6725\\ 0.6889\\ 0.7053\\ 0.7216\\ 0.7380\\ 0.7544\\ 0.7708\\ 0.7872\\ 0.8035\\ 0.8199\\ 0.8363\\ 0.8527\\ 0.8690\\ 0.8854\\ 0.9018\\ 0.9181\\ 0.9345\\ 0.9509\\ 0.9673\\ 0.9836 \end{array}\right)$
			
				\item Al utiliza el método de Gauss-Seidel con la misma tolerancia, se obtienen los siguientes resultados:
				
				n = 25:
				$x_{a} = \left(\begin{array}{c} 0.0385\\ 0.0769\\ 0.1153\\ 0.1539\\ 0.1923\\ 0.2308\\ 0.2693\\ 0.3077\\ 0.3462\\ 0.3847\\ 0.4231\\ 0.4616\\ 0.5001\\ 0.5385\\ 0.5770\\ 0.6154\\ 0.6539\\ 0.6923\\ 0.7308\\ 0.7693\\ 0.8077\\ 0.8462\\ 0.8846\\ 0.9231\\ 0.9615 \end{array}\right)$
				n = 50:
				$x_{a} = \left(\begin{array}{c} 0.0196\\ 0.0392\\ 0.0589\\ 0.0785\\ 0.0981\\ 0.1177\\ 0.1373\\ 0.1570\\ 0.1766\\ 0.1962\\ 0.2158\\ 0.2354\\ 0.2551\\ 0.2747\\ 0.2943\\ 0.3139\\ 0.3335\\ 0.3531\\ 0.3727\\ 0.3924\\ 0.4120\\ 0.4316\\ 0.4512\\ 0.4708\\ 0.4904\\ 0.5100\\ 0.5296\\ 0.5492\\ 0.5688\\ 0.5884\\ 0.6080\\ 0.6276\\ 0.6472\\ 0.6668\\ 0.6864\\ 0.7060\\ 0.7256\\ 0.7452\\ 0.7648\\ 0.7844\\ 0.8040\\ 0.8236\\ 0.8432\\ 0.8628\\ 0.8824\\ 0.9020\\ 0.9216\\ 0.9412\\ 0.9608\\ 0.9804 \end{array}\right)$
				
				
				n = 60:
				$x_{a} = \left(\begin{array}{c} 0.0164\\ 0.0328\\ 0.0492\\ 0.0656\\ 0.0820\\ 0.0985\\ 0.1149\\ 0.1313\\ 0.1477\\ 0.1641\\ 0.1805\\ 0.1969\\ 0.2133\\ 0.2297\\ 0.2461\\ 0.2625\\ 0.2789\\ 0.2953\\ 0.3117\\ 0.3281\\ 0.3445\\ 0.3609\\ 0.3773\\ 0.3937\\ 0.4101\\ 0.4265\\ 0.4429\\ 0.4593\\ 0.4757\\ 0.4921\\ 0.5085\\ 0.5249\\ 0.5413\\ 0.5577\\ 0.5741\\ 0.5905\\ 0.6068\\ 0.6232\\ 0.6396\\ 0.6560\\ 0.6724\\ 0.6888\\ 0.7052\\ 0.7215\\ 0.7379\\ 0.7543\\ 0.7707\\ 0.7871\\ 0.8035\\ 0.8198\\ 0.8362\\ 0.8526\\ 0.8690\\ 0.8853\\ 0.9017\\ 0.9181\\ 0.9345\\ 0.9509\\ 0.9672\\ 0.9836 \end{array}\right)$
				\item Al utilizar el método de Relajación, se encuentran las siguientes soluciones, dependiendo del valor de $w$:
				\begin{itemize}
				\item $w = 0.5$:
				n = 25: $x_{a} = \left(\begin{array}{c} 0.0385\\ 0.0770\\ 0.1155\\ 0.1540\\ 0.1925\\ 0.2310\\ 0.2694\\ 0.3079\\ 0.3464\\ 0.3849\\ 0.4233\\ 0.4618\\ 0.5003\\ 0.5387\\ 0.5772\\ 0.6156\\ 0.6541\\ 0.6925\\ 0.7310\\ 0.7694\\ 0.8078\\ 0.8463\\ 0.8847\\ 0.9231\\ 0.9616 \end{array}\right)$
				
				n = 50: $x_{a} = \left(\begin{array}{c} 0.0197\\ 0.0393\\ 0.0590\\ 0.0787\\ 0.0984\\ 0.1180\\ 0.1377\\ 0.1574\\ 0.1770\\ 0.1967\\ 0.2164\\ 0.2360\\ 0.2557\\ 0.2753\\ 0.2950\\ 0.3146\\ 0.3343\\ 0.3539\\ 0.3735\\ 0.3932\\ 0.4128\\ 0.4324\\ 0.4520\\ 0.4716\\ 0.4913\\ 0.5109\\ 0.5305\\ 0.5501\\ 0.5697\\ 0.5893\\ 0.6088\\ 0.6284\\ 0.6480\\ 0.6676\\ 0.6872\\ 0.7067\\ 0.7263\\ 0.7459\\ 0.7654\\ 0.7850\\ 0.8045\\ 0.8241\\ 0.8436\\ 0.8632\\ 0.8827\\ 0.9023\\ 0.9218\\ 0.9414\\ 0.9609\\ 0.9805 \end{array}\right)$
				
				n = 60: $x_{a} = \left(\begin{array}{c} 0.0165\\ 0.0329\\ 0.0494\\ 0.0659\\ 0.0824\\ 0.0988\\ 0.1153\\ 0.1318\\ 0.1482\\ 0.1647\\ 0.1811\\ 0.1976\\ 0.2141\\ 0.2305\\ 0.2470\\ 0.2634\\ 0.2799\\ 0.2963\\ 0.3127\\ 0.3292\\ 0.3456\\ 0.3620\\ 0.3785\\ 0.3949\\ 0.4113\\ 0.4277\\ 0.4441\\ 0.4605\\ 0.4769\\ 0.4933\\ 0.5097\\ 0.5261\\ 0.5425\\ 0.5589\\ 0.5752\\ 0.5916\\ 0.6080\\ 0.6244\\ 0.6407\\ 0.6571\\ 0.6734\\ 0.6898\\ 0.7061\\ 0.7225\\ 0.7388\\ 0.7552\\ 0.7715\\ 0.7878\\ 0.8042\\ 0.8205\\ 0.8368\\ 0.8531\\ 0.8695\\ 0.8858\\ 0.9021\\ 0.9184\\ 0.9347\\ 0.9511\\ 0.9674\\ 0.9837 \end{array}\right)$
				
				\item $w = 1.6$:	
				n = 25: $x_{a} = \left(\begin{array}{c} 0.0385\\ 0.0770\\ 0.1154\\ 0.1539\\ 0.1924\\ 0.2309\\ 0.2693\\ 0.3078\\ 0.3463\\ 0.3847\\ 0.4232\\ 0.4617\\ 0.5001\\ 0.5386\\ 0.5770\\ 0.6155\\ 0.6540\\ 0.6924\\ 0.7309\\ 0.7693\\ 0.8078\\ 0.8462\\ 0.8847\\ 0.9231\\ 0.9616 \end{array}\right)$
				
				n = 50: $x_{a} = \left(\begin{array}{c} 0.0196\\ 0.0393\\ 0.0589\\ 0.0785\\ 0.0982\\ 0.1178\\ 0.1375\\ 0.1571\\ 0.1767\\ 0.1964\\ 0.2160\\ 0.2356\\ 0.2552\\ 0.2749\\ 0.2945\\ 0.3141\\ 0.3337\\ 0.3534\\ 0.3730\\ 0.3926\\ 0.4122\\ 0.4318\\ 0.4515\\ 0.4711\\ 0.4907\\ 0.5103\\ 0.5299\\ 0.5495\\ 0.5691\\ 0.5887\\ 0.6083\\ 0.6279\\ 0.6475\\ 0.6671\\ 0.6867\\ 0.7063\\ 0.7258\\ 0.7454\\ 0.7650\\ 0.7846\\ 0.8042\\ 0.8238\\ 0.8434\\ 0.8629\\ 0.8825\\ 0.9021\\ 0.9217\\ 0.9413\\ 0.9608\\ 0.9804 \end{array}\right)$
				
				n = 60: $x_{a} =\left(\begin{array}{c} 0.0164\\ 0.0329\\ 0.0493\\ 0.0657\\ 0.0821\\ 0.0986\\ 0.1150\\ 0.1314\\ 0.1479\\ 0.1643\\ 0.1807\\ 0.1971\\ 0.2135\\ 0.2300\\ 0.2464\\ 0.2628\\ 0.2792\\ 0.2956\\ 0.3120\\ 0.3285\\ 0.3449\\ 0.3613\\ 0.3777\\ 0.3941\\ 0.4105\\ 0.4269\\ 0.4433\\ 0.4597\\ 0.4761\\ 0.5089\\ 0.5253\\ 0.5417\\ 0.5580\\ 0.5744\\ 0.5908\\ 0.6072\\ 0.6236\\ 0.6400\\ 0.6563\\ 0.6727\\ 0.6891\\ 0.7055\\ 0.7218\\ 0.7382\\ 0.7546\\ 0.7709\\ 0.7873\\ 0.8037\\ 0.8200\\ 0.8364\\ 0.8528\\ 0.8691\\ 0.8855\\ 0.9018\\ 0.9182\\ 0.9346\\ 0.9509\\ 0.9673\\ 0.9836\\ 0.9872 \end{array}\right)$
				\end{itemize}
			\end{itemize}
			\item Para cada metodo, mencione el criterio de parada y tolerancia, la convergencia (usando el radio espectral), la cantidad de iteraciones, el tiempo usado. Calcule $\frac{||x_{e}-x_{a}||_{\infty}}{||x_{e}||_{\infty}}$. Comente sus resultados.\\
			Para todos los métodos se ocupó una tolerancia de 0.000001 y como criterio de parada se utilizó $\frac{\norm{x^{k+1}-x^{k}}_{2}}{\norm{x^{k+1}}_{2}}\leq 0.000001$. 
			\begin{itemize}
				\item Jacobi:
					\begin{table}[H]
						\centering
						\begin{tabular}{|c|c|c|c|c|}
							\hline 
							n & Iteraciones & Tiempo (s) & Convergencia & Error \\
							\hline
							25 & 1233 &0.66861 & 0.9927 & 0.0080 \\
							\hline
							50 & 4035 & 3.4621 & 0.9927 & 0.0314 \\
							\hline
							60 & 5502 & 5.7737 & 0.9987 & 0.0456 \\
							\hline
						\end{tabular}
					\end{table}
				\item Gauss-Seidel:
				 \begin{table}[H]
						\centering
						\begin{tabular}{|c|c|c|c|c|}
							\hline 
							n & Iteraciones & Tiempo(s) & Convergencia & Error \\
							\hline
							25 & 635 & 0.02531 & 0.9855 & 0.008 \\
							\hline
							50 & 2097 & 0.0539 & 0.9962 & 0.0314 \\
							\hline
							60 & 2867 & 0.0627 & 0.9973 & 0.0456 \\
							\hline
						\end{tabular}
					\end{table}
				\item Relajación:
					\begin{itemize}
					\item $w=0.5$
						\begin{table}[H]
						\centering
						\begin{tabular}{|c|c|c|c|c|}
							\hline 
							n & Iteraciones & Tiempo(s) & Convergencia & Error \\
							\hline
							25 & 2654 & 0.0168 & 0.9971 & 0.0290 \\
							\hline
							50 & 8418 & 0.0516 & 0.9992 & 0.1210 \\
							\hline
							60 & 11365 & 0.0707 & 0.9995 & 0.1824\\
							\hline
						\end{tabular}
					\end{table}
					
					\item $w=1.6$
					\begin{table}[H]
						\centering
						\begin{tabular}{|c|c|c|c|c|}
							\hline 
							n & Iteraciones & Tiempo(s) & Convergencia & Error \\
							\hline
							25 & 1313 & 0.0065 & 0.9935 & 0.0124 \\
							\hline
							50 & 4255 & 0.0247 & 0.9983 & 0.0500 \\
							\hline
							60 & 5784 & 0.0347 & 0.9988 & 0.0732 \\
							\hline
						\end{tabular}
					\end{table}
					\end{itemize}			
			\end{itemize}
 	        
 	        Se observa que la cantidad de iteraciones del método de Jacobi es casi el doble que de Gauss-Seidel, pero la cantidad de tiempo que demora Jacobi es mucho mayor a la de Gauss-Seidel.\\
 	        Al analizar los resultados del método de relajación, se puede observar que el tiempo de convergencia es menor al los métodos mencionado anteriormente, se aprecia que para los diferentes valores de $w$, entre mayor sea $w$, mas rápido converge, y la cantidad de iteraciones disminuye.
			\newpage
 	        \item Si en lugar de usar la forma implementable del método de Gauss-Seidel, se usa la forma general $x^{k+1} = (I-Q^{-I}_{G-S}A)x^k+Q^{-I}_{G-S}b$, cual es la solución del sistema y cual es el tiempo requerido para $n = 25$ y $n = 50$? Compare la solución obtenida en (b) con la que encuentra mediante este proceso.\\
 	        Los resultados encontrados al utilizar la forma general del método de Gauss-Seidel, son:
 	        
 	        n = 25: 
 	       	$x_{a} = \left(\begin{array}{c} 0.0385\\ 0.0769\\ 0.1154\\ 0.1539\\ 0.1924\\ 0.2308\\ 0.2693\\ 0.3078\\ 0.3462\\ 0.3847\\ 0.4232\\ 0.4616\\ 0.5001\\ 0.5385\\ 0.5770\\ 0.6155\\ 0.6539\\ 0.6924\\ 0.7308\\ 0.7693\\ 0.8077\\ 0.8462\\ 0.8846\\ 0.9231\\ 0.9615 \end{array}\right) $
 	        n = 50:
 	        $x_{a} = \left(\begin{array}{c} 0.0196\\ 0.0393\\ 0.0589\\ 0.0785\\ 0.0981\\ 0.1178\\ 0.1374\\ 0.1570\\ 0.1766\\ 0.1963\\ 0.2159\\ 0.2355\\ 0.2551\\ 0.2747\\ 0.2944\\ 0.3140\\ 0.3336\\ 0.3532\\ 0.3728\\ 0.3924\\ 0.4121\\ 0.4317\\ 0.4513\\ 0.4709\\ 0.4905\\ 0.5101\\ 0.5297\\ 0.5493\\ 0.5689\\ 0.5885\\ 0.6081\\ 0.6277\\ 0.6473\\ 0.6669\\ 0.6865\\ 0.7061\\ 0.7257\\ 0.7453\\ 0.7649\\ 0.7845\\ 0.8041\\ 0.8237\\ 0.8433\\ 0.8629\\ 0.8825\\ 0.9021\\ 0.9216\\ 0.9412\\ 0.9608\\ 0.9804 \end{array}\right)$
 	        
 	        Con la información entregada se pudo armar las siguientes tablas:
 	        \begin{itemize}
 	        	\item Para n=25: \begin{table}[H]
 	        			\centering
 	        			\begin{tabular}{|c|c|c|}
 	        				\hline 
							 & General & Implementable \\
							 \hline
							 Tiempo & 0.0253 & 0.0282 \\
							 \hline
							 Iteraciones & 635 & 1233\\
							 \hline
							 Error & 0.0055 & 0.0080\\
							\hline
 	        			\end{tabular}
 	        		\end{table}
 	        		\item Para n=50: \begin{table}[H]
 	        			\centering
 	        			\begin{tabular}{|c|c|c|}
 	        				\hline 
							 & General & Implementable \\
							 \hline
							 Tiempo & 0.0539 & 0.2211 \\
							 \hline
							 Iteraciones & 2097 & 4035 \\
							 \hline
							 Error & 0.0220 & 0.0314\\
							\hline
 	        			\end{tabular}
 	        		\end{table}
 	        		De las tablas se puede extraer que el metodo de Gauss-Seidel en su forma general, es mas eficas a la hora de resolver un sistema, ya que comparando el tiempo de ejecucion, iteraciones y errores, con el metodo de Gauss implementable ,la forma general de Gauss-Seidel es mucho mas efectiva, ya que presenta un menor valor en todo estos aspectos.
 	        \end{itemize}
 	 \end{enumerate}
 \end{enumerate}
 

\newpage
\chapter{Aproximación de funciones} 
\begin{enumerate}
    
\vspace{0.9cm}


\item Utilizando la interpolación por Spline cubico, obtenga las dos curvas que aproximan el contorno del  Snoopy.


\begin{figure}[H]
    \centering
    \includegraphics[width=9cm]{snoopy.png}
    \caption{interpolación Spline cubica.} \label{fig:interp_spline}
\end{figure}


\item Como podría utilizar el polinomio de interpolación de Lagrange para aproximar la parte inferior del contorno del Snoopy? Trace el gráfico que obtiene en esta aproximación.

El polinomio de interpolación de Lagrange se define de la siguiente manera:

 \begin{equation}
 p(x)=\sum_{i=0}^n yi*Li(x)
\end{equation}
\\ Donde Li(x) se define:
 \begin{equation}
 Lk(x)=\prod_{j=0,j\not=k}^{n}  \left( \frac{x-x_{i}}{x_{k}-x_{j}} \right)
\end{equation}
Ya definido el polonimio interpolante de Lagrange, se aplica el algoritmo $pol-lag.m$,algoritmo interpolante de Lagrange, en MATLAB con el fin de obtener la gráfica inferior del Snoopy, utilizando las curvas 4,5 y 6 del problema. 
\begin{figure}[H]
    \centering
    \includegraphics[width=9cm]{snoopy2.png}
    \caption{interpolación Lagrange parte inferior del Snoopy.} \label{fig:snoopy2}
\end{figure}


  
 \chapter{Integración de funciones} 

 \begin{enumerate}
 \item Use la fórmula del trapecio compuesta (para 6 subintervalos) para calcular la fuerza total ejercida por el agua en la cara de la presa.
 \\ 
 \\
 Al usar la formula del trapecio compuesto para 6 sub-intervalos se obtiene una fuerza total ejercida por el agua sobre la presa es de:
 \\
 \\
 f= 2.1658e+09


 \vspace{0.8cm}
 
 \item Usando interpolación de Lagrange, dibuje la curva del lado derecho de la  figura.
 
  \begin{figure}[H]
    \centering
    \includegraphics[width=9cm]{Lagrange.jpg}
    \caption{Curva lado derecho de la presa, interpolación polinómica de Lagrange.} \label{fig:curva_derecha_lagrange}
  
\end{figure}


 \item Usando interpolación por spline, dibuje la curva del lado izquierdo de la  figura.
 \\
 \begin{figure}[H]
    \centering
    \includegraphics[width=9cm]{Spline3.jpg}
    \caption{Curva lado izquierdo de la presa, interpolación por spline cubico.} \label{fig:curvaizq_spline}
\end{figure}
    


 \end{enumerate}
 \end{enumerate}

\newpage

    
    
\chapter{Conclusión}
Finalizando, podemos concluir que estos algoritmos analizados durante el trabajo, permiten resolver una gran variedad de problemas matemáticos, tales como la resolución de sistemas de ecuaciones lineales, aproximación e integración de funciones, los cuales pudimos ver  en concreto el funcionamiento de cada uno de ellos. La aplicación de los algoritmos estudiados son esenciales para la resolución de problemas que comúnmente no podrían ser resueltos de manera convencional, debido a la complejidad de estos. Finalmente este trabajo permitió comprender de manera práctica y fácil la teoría y aplicación de los algoritmos estudiados



\end{document}
